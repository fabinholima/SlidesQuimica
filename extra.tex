% Created 2024-11-11 Mon 07:30
% Intended LaTeX compiler: lualatex
\documentclass{beamer}
\usetheme{default}
\author{fabio}
\date{\today}
\title{}
\hypersetup{
 pdfauthor={fabio},
 pdftitle={},
 pdfkeywords={},
 pdfsubject={},
 pdfcreator={Emacs 29.4 (Org mode 9.6.15)}, 
 pdflang={English}}
\begin{document}

\begin{frame}{Sumário}
\tableofcontents
\end{frame}

\begin{frame}[label={sec:org3040650}]{a}
\end{frame}








\begin{frame}[allowframebreaks]{Regra Final}
\begin{center}
\begin{tabular}{llrl}
\hline
Espécie & Situação & Nox & Exemplos\\[0pt]
\hline
Substância Simples & Qualquer caso & Zero & \ch{H2}, \ch{N2}\\[0pt]
Íon & Qualquer caso & carga íon & \ch{Na^1+}, \ch{Ca^2+}\\[0pt]
Li, Na, K, Rb, Cs e Fr, Ag & Todos os compostos & +1 & NaC\(\ell\), KOH, AgI, NaBr\\[0pt]
Be, Mg, Ca, Sr, Ba, Fr e Zn & Todos os compostos & +2 & \ch{CaC$\ell$2}, MgO, SrS \ch\{\texcolor{red}{Ba}SO4\}, \ch{ZnBr2},\\[0pt]
\ch{A$\ell$3} & Todos os compostos & +3 & \ch{A$\ell$2O3}, \ch{A$\ell$I3}, \$\(\ell\)\$3(OH)3\}\\[0pt]
F (Flúor) & Todos os compostos & -1 & HF, \ch{CF4}, \ch{NF3}, \ch{OF2}\\[0pt]
H (Hidrogênio) &  &  & \\[0pt]
\end{tabular}
\end{center}








\begin{center}
\begin{tabular}{|p{5cm}|c|c|c|}
\hline
Elementos & Situação & Nox & Exemplos\\[0pt]
\hline
Íon & Qualquer caso & carga íon & \ch{Na^1+}, \ch{Ca^2+}\\[0pt]
\hline
Metais Alcalinos  (Li, Na, K, Rb, Cs e Fr) & Em substâncias compostas & + 1 & \ch{H2}, \ch{N2}, C\\[0pt]
\hline
Metais Alcalinos- Terrosos (Be, Mg, Ca, Sr, Ba, e Ra) & Em substâncias compostas & + 2 & \ch{CaC$\ell$2}, MgO, SrS \ch\{\texcolor{red}{Ba}SO4\},\\[0pt]
\hline
Prata: Ag & Em substâncias compostas & +1 & \ch{AgBr},  \ch{Ag2O}\\[0pt]
\hline
Zinco: & Em substâncias compostas & +2 & \ch{ZnBr2}\\[0pt]
\hline
Alumínio:  \ch{A$\ell$} & Em substâncias compostas & +3 & \ch{A$\ell$2O3}, \ch{A$\ell$I3}\\[0pt]
\hline
Enxofre: S & Em sulfetos (quando  S for mais eletronegativo) & -2 & \ch{H2S},  \ch{Na2S},\\[0pt]
\hline
Halogênio (F, Cl, Br, I) & Ligado a ametais (quando o haleto for mais eletronegativo) & -1 & \ch{NaF}, \ch{KBr}\\[0pt]
\hline
Hidrogenio: H & Ligado a ametais (quando o hidrogênio for mais eletronegativo) & +1 & \ch{HBr}, \ch{H2SO4}, \ch{HIO3}\\[0pt]
\hline
Oxigenio: O & Na maioria da substâncias compostas & -2 & \ch{H2SO4}, \ch{HIO3}\\[0pt]
\hline
Peróxido: \ch\{O2\textsuperscript{-2}\} & Alguns compostos & -1 & \ch{H2O2}, \ch{Na2O2}\\[0pt]
\hline
Superóxido: \ch{X2O4}, \ch{YO4} & Alguns compostos & - 1/2 & \ch{CaO4}, \ch{K2O4}\\[0pt]
\hline
\end{tabular}
\end{center}





\begin{table}[htbp]
\caption{\label{tab:org3889c0e}Subdivisões importantes dos hidrocarbonetos}
\begin{supertabular}{BBLB}
\hline
   \cellcolor{green!20} {\bfseries Subgrupo}  &  \cellcolor{green!20} {\bfseries Característica}  &  \cellcolor{green!20} {\bfseries Exemplos}  &  \cellcolor{green!20} {\bfseries Fórmula geral} \\[0pt]
\hline
\{Alcanos $\backslash$\ ou parafinas\} & Cadeia aberta Ligações simples & \chemfig{H_3C-CH_2-CH_2-CH_3} \quad \chemfig{H_3C-C([:90]-CH_3)([:-90]-CH_3)-CH_2-CH([:-90]-CH_3)-CH_3}\qquad & \(\rm C_nH_{2n+2}\)\\[0pt]
\hline
Alcenos, alquenos ou olefinas & Cadeia aberta com 1 ligação dupla & \chemfig{H_2C=CH-CH_2-CH_3} \quad \chemfig{H_3C-C([:90]-CH_3)=CH-CH_3} & \(\rm C_nH_{2n}\)\\[0pt]
\hline
Alcinos ou alquinos & Cadeia aberta 1 ligação tripla & \chemfig{HC~C-CH_2-CH_3} \quad \chemfig{H_3C-C([:90]-CH_3)([:-90]-CH_3)-CH_2-C~C-CH_3} & \(\rm C_nH_{2n-2}\)\\[0pt]
\hline
Alcadienos ou dienos & Cadeia aberta 2 ligações duplas & \chemfig{H_2C=C=CH_2}\qquad \chemfig{H_2C=CH-CH=CH_2} & \(\rm C_nH_{2n-2}\)\\[0pt]
\hline
Ciclanos & Cadeia fechada Ligações simples & \chemfig{*5(-----)} \qquad \chemfig{*6(------)} & \(\rm C_nH_{2n}\)\\[0pt]
\hline
Ciclenos & Cadeia fechada uma ligação dupla & \chemfig{*4(---(-)=)} \qquad \chemfig{*6(-----=)} & \(\rm C_nH_{2n-2}\)\\[0pt]
\hline
Aromáticos & Contêm anel benzênico & \chemfig{**6(-----(-CH_3)-)} \qquad  \chemfig{*6(-=-(*6(-=-=---))=-=)} & \(\rm C_nH_{2n-6}\)\\[0pt]
\end{supertabular}
\end{table}
\end{frame}
\end{document}
