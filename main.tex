% Created 2024-10-13 Sun 18:01
% Intended LaTeX compiler: lualatex
\documentclass[presentation,professionalfonts,aspectratio=169]{beamer}
                

% 
\makeatletter
 \@ifclassloaded{beamer}{%
  %%% save beamer's `solution' environment as `beamersolution':
  \let\beamersolution\solution
  \let\endbeamersolution\endsolution
  %%% "delete" the `solution' environment:
  \let\solution\relax
  \let\endsolution\relax
}{%
}%
\makeatother
\usepackage[utf8]{inputenc}
\usepackage[T1]{fontenc}
%\usepackage[french]{babel}
%\usepackage[portuguese]{babel}

%%%% FONTS




\usepackage{xsim}
\usepackage[most]{tcolorbox}
\usepackage{amssymb}
\usepackage{fontawesome}
\usepackage{tasks}

\newcounter{paragraph}


\DeclareExerciseEnvironmentTemplate{custom}{%
  \begin{tcolorbox}[boxrule = 0pt]
  \tcbox[on line,colback=teal,colframe=teal,coltext=white,size=small]{%
    \faBook\sffamily\bfseries\
    \XSIMmixedcase{\GetExerciseName}
    \GetExerciseProperty{counter}%
  }\quad
}{\end{tcolorbox}}


\DeclareExerciseEnvironmentTemplate{custom2}{%
  \begin{tcolorbox}[boxrule = 0pt]
  \tcbox[on line,colback=violet,colframe=violet,coltext=white,size=small]{%
    \faToggleOn\sffamily\bfseries\
    \XSIMmixedcase{\GetExerciseName}
    \GetExerciseProperty{counter}%
  }\quad
}{\end{tcolorbox}}




\DeclareExerciseType{test}{
	exercise-env = question ,
	solution-env = answer ,
	exercise-template = custom ,
	solution-template = custom2 ,
	exercise-name	= Exemplo,
	exercises-name = Exemplo ,
	solution-name = Solução ,
	solutions-name = Sol. ,
	exercise-heading = \textbf ,
	solution-heading = \textbf
}


\xsimsetup{
  exercise/within = section,
  exercise/the-counter =  \arabic{exercise}, 
%%solution-name = solution,  % used with headings=true
solution/print=false,
print-collection/print=both,
}

\NewTasksEnvironment[label = (\emph{\alph*}), label-width = 12pt]{choice}[\choice]


\usepackage{colortbl}
\usepackage[tikz]{bclogo}
\usetikzlibrary{fit,patterns,shadows.blur,shapes,mindmap}
\usetikzlibrary{arrows,calc,arrows.meta,decorations.markings,shapes.symbols}
\usetikzlibrary{decorations.pathreplacing, decorations.pathmorphing,calc,arrows,positioning}
\usepackage{tikzpeople}
\usepackage{qrcode,hyperref}
\usepackage{upgreek}
%\usepackage[version=4]{mhchem}
\usepackage{tabularray}


\NewTblrTheme{fancy}{
\SetTblrStyle{caption-tag}{font=\bfseries}
\SetTblrInner[tblr,longtblr]{rowsep=2.5pt}
\DefTblrTemplate{firsthead, middlehead,lasthead}{default}{} % <---
\DefTblrTemplate{contfoot-text}{normal}{\scriptsize\textit{Continued on the next page}}
\SetTblrTemplate{contfoot-text}{normal}
}






\usepackage{chemfig,chemmacros,elements,chemformula}
\chemsetup{modules={all}}
\chemsetup[redox]{pos=top,roman=false}
\chemsetup[redox]{pos=top}
\chemsetup{redox/sep=.5em}
\chemsetup[redox]{explicit-sign=true}
\NewChemPhase\lqdd{\(\ell\)}
\NewChemPhase\gr{grafite}
\NewChemPhase\reac{reação}
\NewChemState\Enthalpy{symbol=H,superscript=,unit=\kilo\joule}%
\usepackage{siunitx}
\setchemfig{fixed length=false, atom sep=2.5em, arrow offset=6pt, scheme debug=false}%,angle increment=30}
\renewcommand*\printatom[1]{\ensuremath{\mathsf{#1}}} % This line changes the font of the atoms to sans serif
%%%% QRCODE
\usepackage{pdfpages}
\usepackage{mol2chemfig}
\usepackage{subfig,caption}
\usepackage{wrapfig}
\usepackage{enumitem}
\setitemize{label=\usebeamerfont*{itemize item}%
\usebeamercolor[fg]{itemize item}
\usebeamertemplate {itemize item}}
\usepackage{array} % ajust colunm table
\usepackage{cancel}
\usepackage[controls]{animate}
\renewcommand{\CancelColor}{\color{red}}

%%%%%%%%%%%%%%%%%%% CONFIG TCOLORBOX 

\newtcolorbox{mybox}[2][]{boxsep=0.5em,left=0.5em,
colback=blue!5!white, colframe=blue!75!black,
fonttitle=\bfseries\sffamily,
colbacktitle=blue!85!red!60,enhanced,
attach boxed title to top left={yshift=-3mm,xshift=5mm},
title=#2,#1}

\newtcolorbox{myrule}[2][]{boxsep=0.5em,left=0.5em,
colback=green!5!white, colframe=blue!75!black,
fonttitle=\bfseries\sffamily,
colbacktitle=blue!85!red!60,enhanced,
attach boxed title to top left={yshift=-3mm,xshift=5mm},
title=#2,#1}


\newtcolorbox{myex}[2][]{boxsep=0.5em,left=0.5em,
  colback=yellow!5!white, colframe=blue!75!black, 
  fonttitle=\bfseries\sffamily,
  colbacktitle=blue!85!red!60,enhanced,
  attach boxed title to top left={yshift=-3mm,xshift=5mm},
  title=#2,#1}


 \definecolor{col1}{HTML}{FF7878}
 \definecolor{col2}{HTML}{51B5F8}
 \definecolor{col3}{HTML}{68E1AA}
 \definecolor{col4}{HTML}{B869EA}
 \definecolor{col5}{HTML}{FF5500}
 \definecolor{col6}{HTML}{FFF8E7}
 \definecolor{col7}{HTML}{FF9966}
 \definecolor{col8}{HTML}{9400D3}



\definesubmol\nobond{-[,0.2,,,draw=none]\scriptstyle\color{blue}}
\newcommand{\re}{\hspace{-1cm}}
\newcommand{\af}{\hspace{2cm}}

%%%% Config X sim for BEAMER
\usepackage{ragged2e}
\justifying
\makeatletter
\@ifclassloaded{beamer}{%
%%% save beamer's `solution' environment as `beamersolution':
\let\beamersolution\solution
\let\endbeamersolution\endsolution
%%% "delete" the `solution' environment:
\let\solution\relax
\let\endsolution\relax
}{%
}%
\makeatother
\usepackage[utf8]{inputenc}
\usepackage[T1]{fontenc}
%\usepackage[portuguese, ]{babel}
%%%% FONTS
%%% XSIM CONFIG BEAMER
\usepackage{xsim}
\usepackage[most]{tcolorbox}
\usepackage{amssymb}
\usepackage{fontawesome}
\usepackage{tasks}
\newcounter{paragraph}
%\usepackage[dvipsnames,svgnames]{xcolor}
\usepackage{xcolor}
\usepackage{annotate-equations}
%%% BOX EXERCISE BEAMER
\DeclareExerciseEnvironmentTemplate{custom}{%
\begin{tcolorbox}[boxrule = 0pt]
\tcbox[on line,colback=teal,colframe=teal,coltext=white,size=small]{%
\faBook\sffamily\bfseries\
\XSIMmixedcase{\GetExerciseName}
\GetExerciseProperty{counter}%
}\quad
}{\end{tcolorbox}}
%% == CUSTOM BOX BEAMER
\DeclareExerciseEnvironmentTemplate{custom2}{%
\begin{tcolorbox}[boxrule = 0pt]
\tcbox[on line,colback=violet,colframe=violet,coltext=white,size=small]{%
\faToggleOn\sffamily\bfseries\
\XSIMmixedcase{\GetExerciseName}
\GetExerciseProperty{counter}%
}\quad
}{\end{tcolorbox}}
\DeclareExerciseType{test}{
exercise-env = question ,
solution-env = answer ,
exercise-template = custom ,
solution-template = custom2 ,
exercise-name = Exemplo ,
exercises-name = Exemplo ,
solution-name = Solução ,
solutions-name = Sol. ,
exercise-heading = \textbf ,
solution-heading = \textbf
}
\xsimsetup{
exercise/within = section,
exercise/the-counter =  \arabic{exercise},
%%solution-name = solution,  % used with headings=true
solution/print=true,
print-collection/print=both,
}
\NewTasksEnvironment[label = (\emph{\alph*}),
label-width = 12pt]{choice}[\choice]
\usepackage{empheq} %%% Brackers
\usepackage{colortbl}
\usepackage[tikz]{bclogo}
\usetikzlibrary{fit,patterns,shadows.blur,shapes,mindmap}
\usetikzlibrary{arrows,arrows.meta,decorations.markings,shapes.symbols}
\usetikzlibrary{decorations.pathreplacing, decorations.pathmorphing,calc,arrows,positioning}
\usepackage{tikzpeople}
\usepackage{qrcode,hyperref}
\usepackage{upgreek}
%\usepackage[version=4]{mhchem}
\usepackage{tabularray}
%%% CUSTOM TABLE
\NewTblrTheme{fancy}{
\SetTblrStyle{caption-tag}{font=\bfseries,red2}
\SetTblrInner[tblr,longtblr]{rowsep=2.5pt}
\DefTblrTemplate{firsthead, middlehead,lasthead}{default}{} % <---
\DefTblrTemplate{contfoot-text}{normal}{\scriptsize\textit{Continua ...}}
\SetTblrTemplate{contfoot-text}{normal}
}
%% ==== CHEMMACROS E CHEMFIG CONFIG
\usepackage{chemfig,chemmacros,elements,chemformula}
\chemsetup{modules={all}}
\chemsetup[redox]{pos=top,roman=false}
\chemsetup[redox]{pos=top}
\chemsetup{redox/sep=.5em}
\chemsetup[redox]{explicit-sign=true} %%% reaction redox
%% == CUSTOM PHASES IN CHEMMACROS
\NewChemPhase\lqdd{\(\ell\)}
\NewChemPhase\gr{grafite}
\NewChemPhase\reac{reação}
\NewChemState\Enthalpy{symbol=H,superscript=,unit=\kilo\joule}%
\usepackage{siunitx}
\setchemfig{fixed length=false, atom sep=2.5em, arrow offset=6pt, scheme debug=false}
%% == NUMEROS PARA FORMULES
\renewcommand*\printatom[1]{\ensuremath{\mathsf{#1}}} % This line changes the font of the atoms to sans serif
%%% INCLUDE PAGES PDFs
\usepackage{pdfpages}
\usepackage{mol2chemfig}
\usepackage{subfig,caption}
\usepackage{wrapfig}
\usepackage{enumitem}
\setitemize{label=\usebeamerfont*{itemize item}%
\usebeamercolor[fg]{itemize item}
\usebeamertemplate {itemize item}}
\usepackage{array} % ajust colunm table
\usepackage{cancel}
\usepackage[controls]{animate}
\renewcommand{\CancelColor}{\color{red}}
%%%%%%%%%%%%%%%%%%% CONFIG TCOLORBOX
\newtcolorbox{mybox}[2][]{boxsep=0.5em,left=0.5em,
colback=blue!5!white, colframe=blue!75!black,
fonttitle=\bfseries\sffamily,
colbacktitle=blue!85!red!60,enhanced,
attach boxed title to top left={yshift=-3mm,xshift=5mm},
title=#2,#1}
\newtcolorbox{myrule}[2][]{boxsep=0.5em,left=0.5em,
colback=green!5!white, colframe=blue!75!black,
fonttitle=\bfseries\sffamily,
colbacktitle=blue!85!red!60,enhanced,
attach boxed title to top left={yshift=-3mm,xshift=5mm},
title=#2,#1}
\newtcolorbox{myex}[2][]{boxsep=0.5em,left=0.5em,
colback=yellow!5!white, colframe=blue!75!black,
fonttitle=\bfseries\sffamily,
colbacktitle=blue!85!red!60,enhanced,
attach boxed title to top left={yshift=-3mm,xshift=5mm},
title=#2,#1}
\definecolor{col1}{HTML}{FF7878}
\definecolor{col2}{HTML}{51B5F8}
\definecolor{col3}{HTML}{68E1AA}
\definecolor{col4}{HTML}{B869EA}
\definecolor{col5}{HTML}{FF5500}
\definecolor{col6}{HTML}{FFF8E7}
\definecolor{col7}{HTML}{FF9966}
\definecolor{col8}{HTML}{9400D3}
%% CONFIG COLOR CARBONO
\tikzstyle{bal}=[inner sep=0.3pt,fill=orange,fill opacity=0.5,circle,minimum size=0.2cm]
\tikzstyle{rect}=[inner sep=0.3pt,fill=red,fill opacity=0.5,circle,minimum size=0.2cm]
\tikzstyle{bal2}=[inner sep=0.3pt,fill=blue,fill opacity=0.5,circle,minimum size=0.2cm]
\tikzstyle{bal3}=[inner sep=0.3pt,fill=yellow,fill opacity=0.5,circle,minimum size=0.2cm]
\definesubmol\nobond{-[,0.2,,,draw=none]\scriptstyle\color{blue}}
\newcommand{\re}{\hspace{-1cm}}
\newcommand{\af}{\hspace{2cm}}
\date{}
%\usetheme{minflat}
\usetheme{minflat}
\author{Fábio Lima}
\date{}
\title{ Isomeria Plana}
\hypersetup{
 pdfauthor={Fábio Lima},
 pdftitle={ Isomeria Plana},
 pdfkeywords={},
 pdfsubject={},
 pdfcreator={Emacs 29.4 (Org mode 9.6.15)}, 
 pdflang={En Portuguese}}
\begin{document}

\begingroup
  \setbeamertemplate{headline}{}
  \maketitle
  \endgroup
\begin{frame}{Sumário}
\tableofcontents
\end{frame}


\section{Isomeria Plana}
\label{sec:orgf9b3366}


\begin{frame}[label={sec:org61425b8}]{Isomeria Plana}
\begin{itemize}
\item Pode ser percebida observando-se a  fórmula estrutural plana dos compostos.
\item \alert{Isômeros constitucionais} diferem na  maneira com que seus átomos estão  conectados.
\end{itemize}
\end{frame}


\section{Isomeria de Função}
\label{sec:org465cb39}

\begin{frame}[allowframebreaks]{Isomeria de Função}
\begin{itemize}
\item A diferença entre os isômeros está no grupo funcional.
\end{itemize}



\begin{bclogo}[couleur=yellow!30 , arrondi=0.1 , logo=\bcplume , epBarre=3.5]{Fórmula Molecular: \ch{C3H6O}}


\begin{tblr}
{
colspec = {X[c] X[c]},
colsep = 15mm,
row{2} = {font=\bfseries, fg=teal},
}
\chemname[2ex]{\chemfig{H_3C-\textcolor{red}{C}([2,,,,red]=\textcolor{red}{O})-CH_3}}{Propanona}
& 
\chemname{\chemfig{H_3C-CH_2-\textcolor{red}{C}([1,,,,red]=\textcolor{red}{O})([7,,,,red]-\textcolor{red}{H})}}{Propanal} \\
Cetona & Aldeído \\
\end{tblr}
\end{bclogo}

\framebreak



\begin{bclogo}[couleur=yellow!30 , arrondi=0.1 , logo=\bcplume , epBarre=3.5]{Fórmula Molecular:  \ch{C2H6O}}


\begin{tblr}
{
colspec = {X[c] X[c]},
colsep = 15mm,
row{2} = {font=\bfseries, fg=teal},
}
\chemname[2ex]{\chemfig{H_3C-CH_2-[0,,,,red]\textcolor{red}{OH}}}{Etanol}
& 
\chemname{\chemfig{H_3C-\textcolor{red}{O}-CH_3}}{metóxi-metano} \\
Álcool & Éter \\
\end{tblr}
\end{bclogo}
\end{frame}





\section{Isomeria de cadeia}
\label{sec:orgf3ee7c9}

\begin{frame}[label={sec:org221f0c5}]{Isomeria de cadeia}
\begin{itemize}
\item A diferença entre os isômeros está no tipo de cadeia.
\item Por exemplo, um isômero é de cadeia aberta e o outro de cadeia fechada, ou um é de cadeia normal e o outro de cadeia ramificada, ou então, um tem cadeia homogênea e o outro possui cadeia heterogênea.
\end{itemize}




\begin{bclogo}[couleur=yellow!30 , arrondi=0.1 , logo=\bcplume , epBarre=3.5]{Fórmula Molecular: \ch{C4H10}}


\begin{tblr}
{
colspec = {X[c] X[c]},
colsep = 15mm,
row{2} = {font=\bfseries, fg=teal},
}
\chemname[-2ex]{\chemfig{H_3C-CH([:90]-CH_3)-CH_3}}{Metil-propano}
& 
\chemname[-2ex]{\chemfig{H_3C-CH_2-CH_2-CH_3}}{Butano} \\
Cadeia ramificada & Cadeia normal \\
\end{tblr}
\end{bclogo}
\end{frame}




\section{Isomeria de posição}
\label{sec:orgc78e6a9}

\begin{frame}[label={sec:org645e445}]{Isomeria de posição}
\begin{itemize}
\item A diferença está na posição de uma insaturação, de um grupo funcional, de um heteroátomo ou de um substituinte.
\end{itemize}



\begin{bclogo}[couleur=yellow!30 , arrondi=0.1 , logo=\bcplume , epBarre=3.5]{Fórmula Molecular: \ch{C4H6}}


\begin{tblr}
{
colspec = {X[c] X[c]},
colsep = 15mm,
row{2} = {font=\bfseries, fg=teal},
}
\chemname[-2ex]{\chemfig{HC~[0,,,,red]C-CH_2-CH_3}}{But-1-ino}
& 
\chemname[-2ex]{\chemfig{H_3C-C~[0,,,,red]C-CH_3}}{But-2-ino} \\
Insaturação carbono 1 e 2  & Insaturação carbono 2 e 3 \\
\end{tblr}
\end{bclogo}
\end{frame}




\section{Metameria}
\label{sec:org1b0108e}

\begin{frame}[label={sec:org479a4aa}]{Metameria}
\begin{itemize}
\item É um tipo especial de isomeria de posição, em que a diferença consiste na posição do \alert{heteroátomo}.
\end{itemize}



\begin{bclogo}[couleur=yellow!30 , arrondi=0.1 , logo=\bcplume , epBarre=3.5]{Fórmula Molecular: \ch{C4H6}}


\small{
\begin{tblr}
{
colspec = {X[l] X[r]},
colsep = 15mm,
row{2} = {font=\bfseries, fg=teal},
}
\chemname[2ex]{\chemfig{H_3C-\textcolor{red}{O}-CH_2-CH_2-CH_3}}{Metoxi-propano}
& 
\chemname{\chemfig{H_3C-CH_2-\textcolor{red}{O}-CH_2-CH_3}}{Etoxi-etano} \\
Insaturação carbono 1 e 2  & Insaturação carbono 2 e 3 \\
\end{tblr}
}
\end{bclogo}
\end{frame}




\section{Tautomeria}
\label{sec:org24b81d5}

\begin{frame}[label={sec:orgcbda60b}]{Tautomeria}
\begin{itemize}
\item É um tipo especial de isomeria de função, em que os isômeros coexistem em equilíbrio dinâmico em solução.
\end{itemize}


\begin{bclogo}[couleur=yellow!30 , arrondi=0.1 , logo=\bcplume , epBarre=3.5]{Tautomeria}


\begin{center}
\small
\schemestart
\chemup.
\subscheme{
\chemname[-2ex]{\chemfig{H_3C-\textcolor{red}{C}([2,,,,red]=\textcolor{red}{O})-CH_3}}{Propanona}
\arrow(.mid east--.mid west){<<->}
\chemname[-2ex]{\chemfig{H_2C=[0,,,,red]\textcolor{red}{C}([2,,,,red]-\textcolor{red}{OH})-CH_3}}{propenol}
}
\chemdown\}
\arrow{0}[-90,0]
Equilíbrio ceto-enólico
\schemestop

\vspace{.3cm}

\schemestart
\chemup.
\subscheme{
\chemname[-2ex]{\chemfig{H_3C-\textcolor{red}{C}([1,,,,red]=\textcolor{red}{O})([7,,,,red]-\textcolor{red}{H})}}{Etanal}
\arrow(.mid east--.mid west){<<->}
\chemname[-2ex]{\chemfig{H_3C=[0,,,,red]\textcolor{red}{C}-[0,,,,red]\textcolor{red}{OH}}}{Etenol}
}
\chemdown\}
\arrow{0}[-90,0]
Equilíbrio aldo-enólico
\schemestop
\end{center}
\end{bclogo}
\end{frame}

\section{Isomeria Geometrica}
\label{sec:org0eae0e5}

\begin{frame}[label={sec:org3c4b9ad}]{Isomeria Geométrica}
\begin{itemize}
\item Conhecida como a isomeria \emph{cis-trans}.
\item Ocorre em compostos com dupla ligação ou cíclicos .
\item Compostos com dupla ligação
\end{itemize}
\begin{center}
\begin{tikzpicture}[x=0.75pt,y=0.75pt,yscale=-1,xscale=1]
[   oxygen/.style={circle, ball color=red, minimum size=6mm, inner sep=0},
    hydrogen/.style={circle, ball color=white, minimum size=2.5mm, inner sep=0},
    carbon/.style={circle, ball color=black!75, minimum size=7mm, inner sep=0}
]
\draw   (196.35,141.5) -- (330.17,141.5) -- (272.82,186) -- (139,186) -- cycle ;
\end{tikzpicture}
\end{center}
\end{frame}


\begin{frame}[label={sec:org0d34ca6}]{Exemplos}
\begin{question}
(\alert{UERJ}) O ácido linoleico, essencial à dieta humana, apresenta a seguinte fórmula estrutural espacial:

%\chemfig{-[:330]-[:30]-[:330]-[:30]-[:330]=[:30]-[:330]-[:30]=[:330]-[:30]%
%-[:330]-[:30]-[:330]-[:30]-[:330]-[:30]-[:330](=[:30]O)-[:270]H}
{\scriptsize
\centering 
\chemfig{OH-[:150,,1](=[:90]O)-[:210]-[:150]-[:210]-[:150]-[:210]-[:150]%
-[:210]-[:150]=[:210]-[:270]-[:210]=[:150]-[:90]-[:150]-[:90]-[:150]-[:90]}
}

Como é possível observar, as ligações duplas presentes nos átomos de carbono 9 e 12 afetam o formato espacial da molécula. As conformações espaciais nessas ligações duplas são denominadas, respectivamente:

\begin{choice}(2)
\choice cis e cis
\choice cis e trans
\choice trans e cis
\choice trans e trans
\end{choice}
\end{question}
\end{frame}

\begin{frame}[label={sec:orgc955176}]{}
\begin{answer}
Analisando as duas insaturações das moléculas, observa-se que os ligantes não mostrados são átomos de hidrogênio. Em ambas as insaturações, os átomos de hidrogênio estão do mesmo lado da ligação dupla, logo estão em posição \alert{cis}.

\chemfig{OH-[:150,,1](=[:90]O)-[:210]-[:150]-[:210]-[:150]-[:210]-[:150]%
-[:210]-[:150](-[:90]H)=[:210](-[:150]H)-[:270]-[:210](-[:270]H)=[:150](%
-[:210]H)-[:90]-[:150]-[:90]-[:150]-[:90]}
\end{answer}
\end{frame}
\end{document}
