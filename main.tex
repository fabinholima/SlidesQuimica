% Created 2024-07-21 Sun 20:32
% Intended LaTeX compiler: lualatex
\documentclass[presentation,professionalfonts,smaller,aspectratio=169]{beamer}
                

% 
\makeatletter
 \@ifclassloaded{beamer}{%
  %%% save beamer's `solution' environment as `beamersolution':
  \let\beamersolution\solution
  \let\endbeamersolution\endsolution
  %%% "delete" the `solution' environment:
  \let\solution\relax
  \let\endsolution\relax
}{%
}%
\makeatother
\usepackage[utf8]{inputenc}
\usepackage[T1]{fontenc}
%\usepackage[french]{babel}
\usepackage[portuguese]{babel}

%%%% FONTS




\usepackage{xsim}
\usepackage[most]{tcolorbox}
\usepackage{amssymb}
\usepackage{fontawesome}
\newcounter{paragraph}



\DeclareExerciseEnvironmentTemplate{custom}{%
  \begin{tcolorbox}[boxrule = 0pt]
  \tcbox[on line,colback=teal,colframe=teal,coltext=white,size=small]{%
    \faBook\sffamily\bfseries\
    \XSIMmixedcase{\GetExerciseName}
    \GetExerciseProperty{counter}%
  }\quad
}{\end{tcolorbox}}


\DeclareExerciseEnvironmentTemplate{custom2}{%
  \begin{tcolorbox}[boxrule = 0pt]
  \tcbox[on line,colback=violet,colframe=violet,coltext=white,size=small]{%
    \faToggleOn\sffamily\bfseries\
    \XSIMmixedcase{\GetExerciseName}
    \GetExerciseProperty{counter}%
  }\quad
}{\end{tcolorbox}}




\DeclareExerciseType{test}{
	exercise-env = question ,
	solution-env = answer ,
	exercise-template = custom ,
	solution-template = custom2 ,
	exercise-name	= Exemplo. ,
	exercises-name = Exemplo ,
	solution-name = Solução ,
	solutions-name = Sol. ,
	exercise-heading = \textbf ,
	solution-heading = \textbf
}


\xsimsetup{
  exercise/within = section,
  exercise/the-counter =  \arabic{exercise}, 
%%solution-name = solution,  % used with headings=true
solution/print=false,
%print-collection/print=both,
}





\usepackage{colortbl}
\usepackage[tikz]{bclogo}
\usetikzlibrary{fit,patterns,shadows.blur,shapes,mindmap}
\usetikzlibrary{arrows,calc,arrows.meta,decorations.markings,shapes.symbols}
\usetikzlibrary{decorations.pathreplacing, decorations.pathmorphing,calc,arrows,positioning}
\usepackage{tikzpeople}
\usepackage{qrcode,hyperref}
\usepackage{upgreek}
%\usepackage[version=4]{mhchem}
\usepackage{tabularray}


\NewTblrTheme{fancy}{
\SetTblrStyle{caption-tag}{font=\bfseries}
\SetTblrInner[tblr,longtblr]{rowsep=2.5pt}
\DefTblrTemplate{firsthead, middlehead,lasthead}{default}{} % <---
\DefTblrTemplate{contfoot-text}{normal}{\scriptsize\textit{Continued on the next page}}
\SetTblrTemplate{contfoot-text}{normal}
}






\usepackage{chemfig,chemmacros,elements,chemformula}
\chemsetup{modules={all}}
\chemsetup[redox]{pos=top,roman=false}
\chemsetup[redox]{pos=top}
\chemsetup{redox/sep=.5em}
\chemsetup[redox]{explicit-sign=true}
\NewChemPhase\lqdd{\(\ell\)}
\NewChemPhase\gr{grafite}
\NewChemPhase\reac{reação}
\NewChemState\Enthalpy{symbol=H,superscript=,unit=\kilo\joule}%
\usepackage{siunitx}
\setchemfig{fixed length=false, atom sep=2.5em, arrow offset=6pt, scheme debug=false}%,angle increment=30}
\renewcommand*\printatom[1]{\ensuremath{\mathsf{#1}}} % This line changes the font of the atoms to sans serif
%%%% QRCODE
\usepackage{pdfpages}
\usepackage{mol2chemfig}
\usepackage{subfig,caption}
\usepackage{wrapfig}
\usepackage{enumitem}
\setitemize{label=\usebeamerfont*{itemize item}%
\usebeamercolor[fg]{itemize item}
\usebeamertemplate {itemize item}}
\usepackage{array} % ajust colunm table
\usepackage{cancel}
\usepackage[controls]{animate}
\renewcommand{\CancelColor}{\color{red}}

%%%%%%%%%%%%%%%%%%% CONFIG TCOLORBOX 

\newtcolorbox{mybox}[2][]{boxsep=0.5em,left=0.5em,
colback=blue!5!white, colframe=blue!75!black,
fonttitle=\bfseries\sffamily,
colbacktitle=blue!85!red!60,enhanced,
attach boxed title to top left={yshift=-3mm,xshift=5mm},
title=#2,#1}

\newtcolorbox{myrule}[2][]{boxsep=0.5em,left=0.5em,
colback=green!5!white, colframe=blue!75!black,
fonttitle=\bfseries\sffamily,
colbacktitle=blue!85!red!60,enhanced,
attach boxed title to top left={yshift=-3mm,xshift=5mm},
title=#2,#1}


\newtcolorbox{myex}[2][]{boxsep=0.5em,left=0.5em,
  colback=yellow!5!white, colframe=blue!75!black, 
  fonttitle=\bfseries\sffamily,
  colbacktitle=blue!85!red!60,enhanced,
  attach boxed title to top left={yshift=-3mm,xshift=5mm},
  title=#2,#1}


 \definecolor{col1}{HTML}{FF7878}
 \definecolor{col2}{HTML}{51B5F8}
 \definecolor{col3}{HTML}{68E1AA}
 \definecolor{col4}{HTML}{B869EA}
 \definecolor{col5}{HTML}{FF5500}
 \definecolor{col6}{HTML}{FFF8E7}
 \definecolor{col7}{HTML}{FF9966}
 \definecolor{col8}{HTML}{9400D3}



\definesubmol\nobond{-[,0.2,,,draw=none]\scriptstyle\color{blue}}
\newcommand{\re}{\hspace{-1cm}}
\newcommand{\af}{\hspace{2cm}}

%%%% Config X sim for BEAMER
\makeatletter
\@ifclassloaded{beamer}{%
%%% save beamer's `solution' environment as `beamersolution':
\let\beamersolution\solution
\let\endbeamersolution\endsolution
%%% "delete" the `solution' environment:
\let\solution\relax
\let\endsolution\relax
}{%
}%
\makeatother
\usepackage[utf8]{inputenc}
\usepackage[T1]{fontenc}
\usepackage[portuguese, ]{babel}
%%%% FONTS
%%% XSIM CONFIG BEAMER
\usepackage{xsim}
\usepackage[most]{tcolorbox}
\usepackage{amssymb}
\usepackage{fontawesome}
\newcounter{paragraph}
%%% BOX EXERCISE BEAMER
\DeclareExerciseEnvironmentTemplate{custom}{%
\begin{tcolorbox}[boxrule = 0pt]
\tcbox[on line,colback=teal,colframe=teal,coltext=white,size=small]{%
\faBook\sffamily\bfseries\
\XSIMmixedcase{\GetExerciseName}
\GetExerciseProperty{counter}%
}\quad
}{\end{tcolorbox}}
%% == CUSTOM BOX BEAMER
\DeclareExerciseEnvironmentTemplate{custom2}{%
\begin{tcolorbox}[boxrule = 0pt]
\tcbox[on line,colback=violet,colframe=violet,coltext=white,size=small]{%
\faToggleOn\sffamily\bfseries\
\XSIMmixedcase{\GetExerciseName}
\GetExerciseProperty{counter}%
}\quad
}{\end{tcolorbox}}
\DeclareExerciseType{test}{
exercise-env = question ,
solution-env = answer ,
exercise-template = custom ,
solution-template = custom2 ,
exercise-name = Exemplo. ,
exercises-name = Exemplo ,
solution-name = Solução ,
solutions-name = Sol. ,
exercise-heading = \textbf ,
solution-heading = \textbf
}
\xsimsetup{
exercise/within = section,
exercise/the-counter =  \arabic{exercise},
%%solution-name = solution,  % used with headings=true
solution/print=false,
%print-collection/print=both,
}
\usepackage{colortbl}
\usepackage[tikz]{bclogo}
\usetikzlibrary{fit,patterns,shadows.blur,shapes,mindmap}
\usetikzlibrary{arrows,calc,arrows.meta,decorations.markings,shapes.symbols}
\usetikzlibrary{decorations.pathreplacing, decorations.pathmorphing,calc,arrows,positioning}
\usepackage{tikzpeople}
\usepackage{qrcode,hyperref}
\usepackage{upgreek}
%\usepackage[version=4]{mhchem}
\usepackage{tabularray}
%%% CUSTOM TABLE
\NewTblrTheme{fancy}{
\SetTblrStyle{caption-tag}{font=\bfseries}
\SetTblrInner[tblr,longtblr]{rowsep=2.5pt}
\DefTblrTemplate{firsthead, middlehead,lasthead}{default}{} % <---
\DefTblrTemplate{contfoot-text}{normal}{\scriptsize\textit{Continua ...}}
\SetTblrTemplate{contfoot-text}{normal}
}
%% ==== CHEMMACROS E CHEMFIG CONFIG
\usepackage{chemfig,chemmacros,elements,chemformula}
\chemsetup{modules={all}}
\chemsetup[redox]{pos=top,roman=false}
\chemsetup[redox]{pos=top}
\chemsetup{redox/sep=.5em}
\chemsetup[redox]{explicit-sign=true} %%% reaction redox
%% == CUSTOM PHASES IN CHEMMACROS
\NewChemPhase\lqdd{\(\ell\)}
\NewChemPhase\gr{grafite}
\NewChemPhase\reac{reação}
\NewChemState\Enthalpy{symbol=H,superscript=,unit=\kilo\joule}%
\usepackage{siunitx}
\setchemfig{fixed length=false, atom sep=2.5em, arrow offset=6pt, scheme debug=false}
%% == NUMEROS PARA FORMULES
\renewcommand*\printatom[1]{\ensuremath{\mathsf{#1}}} % This line changes the font of the atoms to sans serif
%%% INCLUDE PAGES PDFs
\usepackage{pdfpages}
\usepackage{mol2chemfig}
\usepackage{subfig,caption}
\usepackage{wrapfig}
\usepackage{enumitem}
\setitemize{label=\usebeamerfont*{itemize item}%
\usebeamercolor[fg]{itemize item}
\usebeamertemplate {itemize item}}
\usepackage{array} % ajust colunm table
\usepackage{cancel}
\usepackage[controls]{animate}
\renewcommand{\CancelColor}{\color{red}}
%%%%%%%%%%%%%%%%%%% CONFIG TCOLORBOX
\newtcolorbox{mybox}[2][]{boxsep=0.5em,left=0.5em,
colback=blue!5!white, colframe=blue!75!black,
fonttitle=\bfseries\sffamily,
colbacktitle=blue!85!red!60,enhanced,
attach boxed title to top left={yshift=-3mm,xshift=5mm},
title=#2,#1}
\newtcolorbox{myrule}[2][]{boxsep=0.5em,left=0.5em,
colback=green!5!white, colframe=blue!75!black,
fonttitle=\bfseries\sffamily,
colbacktitle=blue!85!red!60,enhanced,
attach boxed title to top left={yshift=-3mm,xshift=5mm},
title=#2,#1}
\newtcolorbox{myex}[2][]{boxsep=0.5em,left=0.5em,
colback=yellow!5!white, colframe=blue!75!black,
fonttitle=\bfseries\sffamily,
colbacktitle=blue!85!red!60,enhanced,
attach boxed title to top left={yshift=-3mm,xshift=5mm},
title=#2,#1}
\definecolor{col1}{HTML}{FF7878}
\definecolor{col2}{HTML}{51B5F8}
\definecolor{col3}{HTML}{68E1AA}
\definecolor{col4}{HTML}{B869EA}
\definecolor{col5}{HTML}{FF5500}
\definecolor{col6}{HTML}{FFF8E7}
\definecolor{col7}{HTML}{FF9966}
\definecolor{col8}{HTML}{9400D3}
\definesubmol\nobond{-[,0.2,,,draw=none]\scriptstyle\color{blue}}
\newcommand{\re}{\hspace{-1cm}}
\newcommand{\af}{\hspace{2cm}}
\date{}
% \usetheme{minflat}
\usetheme{minflat}
\author{Fábio Lima}
\date{}
\title{Hidrocarbonetos Ramificados}
\hypersetup{
 pdfauthor={Fábio Lima},
 pdftitle={Hidrocarbonetos Ramificados},
 pdfkeywords={},
 pdfsubject={},
 pdfcreator={Emacs 29.4 (Org mode 9.6.15)}, 
 pdflang={En Portuguese}}
\begin{document}

\begingroup
  \setbeamertemplate{headline}{}
  \maketitle
  \endgroup
\begin{frame}{Sumário}
\tableofcontents
\end{frame}




\section{Alcanos}
\label{sec:org25ac3b2}

\begin{frame}[label={sec:org575c06d}]{Hidrocarbonetos não-ramificados}
\begin{block}{Alcanos}
\begin{columns}
\begin{column}{0.5\textwidth}
%\setchemfig{angle increment=30}
 \centering
 \vspace{.5cm}
 \chemfig{H_3C-CH_2-CH_2-CH_3}
\\  \vspace{.5cm} ou  \vspace{.5cm} \\
 \chemfig{-[1]-[-1]-[1]} \\  \vspace{.5cm}
 \end{column}
\begin{column}{0.5\textwidth}  %%<--- here
    \begin{center}
   \begin{tikzpicture}[x=0.75pt,y=0.75pt,yscale=-0.7,xscale=.7]
	\draw    (69.73,58.47) -- (70,81) ;
	\draw    (70,81) -- (96.13,80.63) ;
	\draw [shift={(98.13,80.6)}, rotate = 539.19] [color={rgb, 255:red, 0; green, 0; blue, 0 }  ][line width=0.75]    (10.93,-3.29) .. controls (6.95,-1.4) and (3.31,-0.3) .. (0,0) .. controls (3.31,0.3) and (6.95,1.4) .. (10.93,3.29)   ;
	\draw    (50.77,57.43) -- (50.59,83.04) -- (50.93,103.27) ;
	\draw    (50.93,103.27) -- (83.33,103.27) ;
	\draw [shift={(85.33,103.27)}, rotate = 180] [color={rgb, 255:red, 0; green, 0; blue, 0 }  ][line width=0.75]    (7.65,-2.3) .. controls (4.86,-0.97) and (2.31,-0.21) .. (0,0) .. controls (2.31,0.21) and (4.86,0.98) .. (7.65,2.3)   ;
	\draw    (25.73,56.47) -- (26.27,99) -- (26.53,129.47) ;
	\draw    (26.53,129.47) -- (69.33,129.47) ;
	\draw [shift={(71.33,129.47)}, rotate = 180] [color={rgb, 255:red, 0; green, 0; blue, 0 }  ][line width=0.75]    (7.65,-2.3) .. controls (4.86,-0.97) and (2.31,-0.21) .. (0,0) .. controls (2.31,0.21) and (4.86,0.98) .. (7.65,2.3)   ;
	
	% Text Node
	\draw (9.6,34.8) node [anchor=north west][inner sep=0.75pt]   [align=left] {{\small But \ an o}};
	% Text Node
	\draw (102.6,67.6) node [anchor=north west][inner sep=0.75pt]   [align=left] {{\small hidrocarboneto}};
	% Text Node
	\draw (90,95.2) node [anchor=north west][inner sep=0.75pt]   [align=left] {{\small Ligação simples entre carbono}};
	% Text Node
	\draw (88.4,117.6) node [anchor=north west][inner sep=0.75pt]   [align=left] {{\small 4 carbonos}};
	
\end{tikzpicture}

     \end{center}
\end{column}
\end{columns}
%%%%%%%%%%%%%%%% Colunm 2
\rule{14cm}{.3pt}
\begin{columns}
\begin{column}{0.5\textwidth}
%\setchemfig{angle increment=30}
 \centering
\\ \vspace{.5cm} 
 \chemfig{H_3C-CH_2-CH_2-CH_2-CH_3}
\\ \vspace{.5cm} ou \\ \vspace{.5cm}
 \chemfig{-[1]-[-1]-[1]-[-1]} \\
 
\end{column}
\begin{column}{0.5\textwidth}  %%<--- here
    \begin{center}
   \begin{tikzpicture}[x=0.75pt,y=0.75pt,yscale=-.7,xscale=.7]
	\draw    (69.73,58.47) -- (70,81) ;
	\draw    (70,81) -- (96.13,80.63) ;
	\draw [shift={(98.13,80.6)}, rotate = 539.19] [color={rgb, 255:red, 0; green, 0; blue, 0 }  ][line width=0.75]    (10.93,-3.29) .. controls (6.95,-1.4) and (3.31,-0.3) .. (0,0) .. controls (3.31,0.3) and (6.95,1.4) .. (10.93,3.29)   ;
	\draw    (50.77,57.43) -- (50.59,83.04) -- (50.93,103.27) ;
	\draw    (50.93,103.27) -- (83.33,103.27) ;
	\draw [shift={(85.33,103.27)}, rotate = 180] [color={rgb, 255:red, 0; green, 0; blue, 0 }  ][line width=0.75]    (7.65,-2.3) .. controls (4.86,-0.97) and (2.31,-0.21) .. (0,0) .. controls (2.31,0.21) and (4.86,0.98) .. (7.65,2.3)   ;
	\draw    (25.73,56.47) -- (26.27,99) -- (26.53,129.47) ;
	\draw    (26.53,129.47) -- (69.33,129.47) ;
	\draw [shift={(71.33,129.47)}, rotate = 180] [color={rgb, 255:red, 0; green, 0; blue, 0 }  ][line width=0.75]    (7.65,-2.3) .. controls (4.86,-0.97) and (2.31,-0.21) .. (0,0) .. controls (2.31,0.21) and (4.86,0.98) .. (7.65,2.3)   ;
	
	% Text Node
	\draw (9.6,34.8) node [anchor=north west][inner sep=0.75pt]   [align=left] {{\small Pent \ an o}};
	% Text Node
	\draw (102.6,67.6) node [anchor=north west][inner sep=0.75pt]   [align=left] {{\small hidrocarboneto}};
	% Text Node
	\draw (90,95.2) node [anchor=north west][inner sep=0.75pt]   [align=left] {{\small Ligação simples entre carbono}};
	% Text Node
	\draw (88.4,117.6) node [anchor=north west][inner sep=0.75pt]   [align=left] {{\small 5 carbonos}};
	
\end{tikzpicture}

     \end{center}
\end{column}
\end{columns}
\end{block}
\end{frame}




\section{Alcenos}
\label{sec:orgfb91c98}

\begin{frame}[label={sec:org9c5290c}]{Hidrocarbonetos não-ramificados}
\begin{block}{Alcenos}
\begin{columns}
\begin{column}{0.5\textwidth}
%\setchemfig{angle increment=30}
 \centering
 \vspace{.5cm}
 \chemfig{H_2C=CH_2}
\\  \vspace{.5cm}% ou  \vspace{.5cm} \\
% \chemfig{-[1]-[-1]-[1]} \\  \vspace{.5cm}
 \end{column}
\begin{column}{0.5\textwidth}  %%<--- here
    \begin{center}
   \begin{tikzpicture}[x=0.75pt,y=0.75pt,yscale=-0.7,xscale=.7]
	\draw    (69.73,58.47) -- (70,81) ;
	\draw    (70,81) -- (96.13,80.63) ;
	\draw [shift={(98.13,80.6)}, rotate = 539.19] [color={rgb, 255:red, 0; green, 0; blue, 0 }  ][line width=0.75]    (10.93,-3.29) .. controls (6.95,-1.4) and (3.31,-0.3) .. (0,0) .. controls (3.31,0.3) and (6.95,1.4) .. (10.93,3.29)   ;
	\draw    (50.77,57.43) -- (50.59,83.04) -- (50.93,103.27) ;
	\draw    (50.93,103.27) -- (83.33,103.27) ;
	\draw [shift={(85.33,103.27)}, rotate = 180] [color={rgb, 255:red, 0; green, 0; blue, 0 }  ][line width=0.75]    (7.65,-2.3) .. controls (4.86,-0.97) and (2.31,-0.21) .. (0,0) .. controls (2.31,0.21) and (4.86,0.98) .. (7.65,2.3)   ;
	\draw    (25.73,56.47) -- (26.27,99) -- (26.53,129.47) ;
	\draw    (26.53,129.47) -- (69.33,129.47) ;
	\draw [shift={(71.33,129.47)}, rotate = 180] [color={rgb, 255:red, 0; green, 0; blue, 0 }  ][line width=0.75]    (7.65,-2.3) .. controls (4.86,-0.97) and (2.31,-0.21) .. (0,0) .. controls (2.31,0.21) and (4.86,0.98) .. (7.65,2.3)   ;
	
	% Text Node
	\draw (9.6,34.8) node [anchor=north west][inner sep=0.75pt]   [align=left] {{\small et \ en o}};
	% Text Node
	\draw (102.6,67.6) node [anchor=north west][inner sep=0.75pt]   [align=left] {{\small hidrocarboneto}};
	% Text Node
	\draw (90,95.2) node [anchor=north west][inner sep=0.75pt]   [align=left] {{\small Ligação dupla entre carbono}};
	% Text Node
	\draw (88.4,117.6) node [anchor=north west][inner sep=0.75pt]   [align=left] {{\small 2 carbonos}};
	
\end{tikzpicture}

     \end{center}
\end{column}
\end{columns}
%%%%%%%%%%%%%%%% Colunm 2
\rule{14cm}{.3pt}
\begin{columns}
\begin{column}{0.5\textwidth}
%\setchemfig{angle increment=30}
 \centering
 \vspace{.5cm} 
\chemfig{H_2C=CH-CH_3}
%\chemfig{H_3C-CH_2-CH_2-CH_2-CH_3}
 \vspace{.5cm} ou \\ \vspace{.5cm}
\chemfig{CH_3-CH=CH_2}\\
\alert{É a mesma molécula, porém escrita de modo diferentes}
% \chemfig{-[1]-[-1]-[1]-[-1]} \\
 
\end{column}
\begin{column}{0.5\textwidth}  %%<--- here
    \begin{center}
   \begin{tikzpicture}[x=0.75pt,y=0.75pt,yscale=-.7,xscale=.7]
	\draw    (69.73,58.47) -- (70,81) ;
	\draw    (70,81) -- (96.13,80.63) ;
	\draw [shift={(98.13,80.6)}, rotate = 539.19] [color={rgb, 255:red, 0; green, 0; blue, 0 }  ][line width=0.75]    (10.93,-3.29) .. controls (6.95,-1.4) and (3.31,-0.3) .. (0,0) .. controls (3.31,0.3) and (6.95,1.4) .. (10.93,3.29)   ;
	\draw    (50.77,57.43) -- (50.59,83.04) -- (50.93,103.27) ;
	\draw    (50.93,103.27) -- (83.33,103.27) ;
	\draw [shift={(85.33,103.27)}, rotate = 180] [color={rgb, 255:red, 0; green, 0; blue, 0 }  ][line width=0.75]    (7.65,-2.3) .. controls (4.86,-0.97) and (2.31,-0.21) .. (0,0) .. controls (2.31,0.21) and (4.86,0.98) .. (7.65,2.3)   ;
	\draw    (25.73,56.47) -- (26.27,99) -- (26.53,129.47) ;
	\draw    (26.53,129.47) -- (69.33,129.47) ;
	\draw [shift={(71.33,129.47)}, rotate = 180] [color={rgb, 255:red, 0; green, 0; blue, 0 }  ][line width=0.75]    (7.65,-2.3) .. controls (4.86,-0.97) and (2.31,-0.21) .. (0,0) .. controls (2.31,0.21) and (4.86,0.98) .. (7.65,2.3)   ;
	
	% Text Node
	\draw (9.6,34.8) node [anchor=north west][inner sep=0.75pt]   [align=left] {{\small prop \ en o}};
	% Text Node
	\draw (102.6,67.6) node [anchor=north west][inner sep=0.75pt]   [align=left] {{\small hidrocarboneto}};
	% Text Node
	\draw (90,95.2) node [anchor=north west][inner sep=0.75pt]   [align=left] {{\small Ligação dupla entre carbono}};
	% Text Node
	\draw (88.4,117.6) node [anchor=north west][inner sep=0.75pt]   [align=left] {{\small 3 carbonos}};
	
\end{tikzpicture}

     \end{center}
\end{column}
\end{columns}
\end{block}
\end{frame}



\begin{frame}[label={sec:org32356b7}]{Hidrocarbonetos não-ramificados}
\framesubtitle{Alcenos}
\begin{columns}
\begin{column}{0.45\columnwidth}
\begin{block}{Numeração correta}
\vspace{.5cm}

\chemfig{H_3\mcfabove{C}{\mcfatomno{4}}-\mcfabove{C}{\mcfatomno{3}}H_2-\mcfabove{C}{\mcfatomno{2}}H=\mcfabove{C}{\mcfatomno{1}}H_2}


\vspace{.5cm}

Nome correto: \alert{but-1-eno} 

\vspace{.5cm}

Extremidade mais próxima da insaturação
\end{block}
\end{column}

\begin{column}{0.45\columnwidth}
\begin{block}{Numeração incorreta}
\vspace{.5cm}

\chemfig{H_3\mcfabove{C}{\mcfatomno{1}}-\mcfabove{C}{\mcfatomno{2}}H_2-\mcfabove{C}{\mcfatomno{3}}H=\mcfabove{C}{\mcfatomno{4}}H_2}

\vspace{.5cm}

Nome incorreto: \emph{but-3-eno}

\vspace{.5cm}

Extremidade mais próxima da insaturação
\end{block}
\end{column}
\end{columns}
\end{frame}



\begin{frame}[label={sec:org790b912}]{Hidrocarbonetos não-ramificados}
\framesubtitle{Alcenos}
\begin{columns}
\begin{column}{0.45\columnwidth}
\begin{block}{Numeração correta}
\vspace{.5cm}

\chemfig{H_3\mcfabove{C}{\mcfatomno{4}}-\mcfabove{C}{\mcfatomno{3}}H_2-\mcfabove{C}{\mcfatomno{2}}H=\mcfabove{C}{\mcfatomno{1}}H_2}


\vspace{.5cm}

Nome correto: \alert{but-1-eno} 

\vspace{.5cm}

Extremidade mais próxima da insaturação
\end{block}
\end{column}

\begin{column}{0.45\columnwidth}
\begin{block}{Numeração correta}
\vspace{.5cm}

\chemfig{H_3\mcfabove{C}{\mcfatomno{1}}-\mcfabove{C}{\mcfatomno{2}}H=\mcfabove{C}{\mcfatomno{3}}H-\mcfabove{C}{\mcfatomno{4}}H_3}

\vspace{.5cm}

Nome correto: \alert{but-2-eno}

\vspace{.5cm}

Posição da dupla ligação difere entre as moléculas
\end{block}
\end{column}
\end{columns}
\end{frame}


\section{Alcinos}
\label{sec:org4349d46}

\begin{frame}[label={sec:org22d9f66}]{Hidrocarbonetos não-ramificados}
\begin{block}{Alcinos}
\begin{columns}
\begin{column}{0.5\textwidth}
%\setchemfig{angle increment=30}
 \centering
 \vspace{.5cm}
 \chemfig{HC~CH_2}
\\  \vspace{.5cm}% ou  \vspace{.5cm} \\
% \chemfig{-[1]-[-1]-[1]} \\  \vspace{.5cm}
 \end{column}
\begin{column}{0.5\textwidth}  %%<--- here
    \begin{center}
   \begin{tikzpicture}[x=0.75pt,y=0.75pt,yscale=-0.7,xscale=.7]
	\draw    (69.73,58.47) -- (70,81) ;
	\draw    (70,81) -- (96.13,80.63) ;
	\draw [shift={(98.13,80.6)}, rotate = 539.19] [color={rgb, 255:red, 0; green, 0; blue, 0 }  ][line width=0.75]    (10.93,-3.29) .. controls (6.95,-1.4) and (3.31,-0.3) .. (0,0) .. controls (3.31,0.3) and (6.95,1.4) .. (10.93,3.29)   ;
	\draw    (50.77,57.43) -- (50.59,83.04) -- (50.93,103.27) ;
	\draw    (50.93,103.27) -- (83.33,103.27) ;
	\draw [shift={(85.33,103.27)}, rotate = 180] [color={rgb, 255:red, 0; green, 0; blue, 0 }  ][line width=0.75]    (7.65,-2.3) .. controls (4.86,-0.97) and (2.31,-0.21) .. (0,0) .. controls (2.31,0.21) and (4.86,0.98) .. (7.65,2.3)   ;
	\draw    (25.73,56.47) -- (26.27,99) -- (26.53,129.47) ;
	\draw    (26.53,129.47) -- (69.33,129.47) ;
	\draw [shift={(71.33,129.47)}, rotate = 180] [color={rgb, 255:red, 0; green, 0; blue, 0 }  ][line width=0.75]    (7.65,-2.3) .. controls (4.86,-0.97) and (2.31,-0.21) .. (0,0) .. controls (2.31,0.21) and (4.86,0.98) .. (7.65,2.3)   ;
	
	% Text Node
	\draw (9.6,34.8) node [anchor=north west][inner sep=0.75pt]   [align=left] {{\small et\ \alert{in} o}};
	% Text Node
	\draw (102.6,67.6) node [anchor=north west][inner sep=0.75pt]   [align=left] {{\small hidrocarboneto}};
	% Text Node
	\draw (90,95.2) node [anchor=north west][inner sep=0.75pt]   [align=left] {{\small Ligação tripla entre carbono}};
	% Text Node
	\draw (88.4,117.6) node [anchor=north west][inner sep=0.75pt]   [align=left] {{\small 2 carbonos}};
	
\end{tikzpicture}

     \end{center}
\end{column}
\end{columns}

\begin{bclogo}[logo=\bcattention, noborder=true, barre=none]{Atenção}
Quando houver mais de uma possibilidade para a localização da insaturação, deve-se indicar sua posição de modo similar ao que foi feito no caso dos alcenos.

\begin{center}
\begin{tabular}{ll}
\chemfig{HC~CH_2-CH_3} & \alert{but-1-ino}\\[0pt]
\chemfig{H_3C-C~C-CH_3} & but-2-ino\\[0pt]
\chemfig{H_3C-CH_2-C~CH} & \alert{but-1-ino}\\[0pt]
\end{tabular}
\end{center}

\begin{itemize}
\item No caso da estrutura do \alert{but-1-ino} é a mesma molécula.
\end{itemize}
\end{bclogo}
\end{block}
\end{frame}


\section{Alcadienos}
\label{sec:orge8b9af5}

\begin{frame}[label={sec:orga74ac71}]{Dienos}
\begin{bclogo}[logo=\bcinfo, noborder=true, barre=none]{Exemplo}

\begin{columns}
\begin{column}{0.4\textwidth}
%\vspace{.3cm}
\schemestart 
\chemname{
\chemfig{H_2C=C=CH-CH_3}}{\small posição das duplas}
\chemmove{
\node[text width=3cm,blue] at (2.0 ,0) (A) {buta-1,2-dieno};
\draw[|->] (2.3,-.1)--(2.3,-0.8); % Line 1
\node[text width=2.0cm,blue] at (2.8 ,-1.1) (A) {\scriptsize \emph{di} duas e \emph{en} (dupla ligação)};
\draw[|->] (1.5,-.1)--(1.5,-0.5)--(-0.28,-.5); % seta dupla
\draw[|->] (1.1, -0.1)--(1.1,-1.9); % seta do A
\node[text width=3cm,blue] at (1.5 ,-2.2) {\scriptsize Note a presença do \emph{a}};
}
\schemestop

\end{column}
\begin{column}{0.5\textwidth}

%\vspace{.3cm}
\schemestart 
\chemname{
\chemfig{H_2C=C=CH-CH_2-CH_3}}{\small posição das duplas}
\chemmove{
\node[text width=3cm,blue] at (2.0 ,0) (A) {penta-1,2-dieno};
\draw[|->] (2.4,-.1)--(2.4,-0.8); % Line 1
\node[text width=2.0cm,blue] at (2.8 ,-1.1) (A) {\scriptsize \emph{di} duas e \emph{en} (dupla ligação)};
\draw[|->] (1.7,-.1)--(1.7,-0.5)--(-0.28,-.5); % seta dupla
\draw[|->] (1.2, -0.1)--(1.2,-1.9); % seta do A
\node[text width=3cm,blue] at (1.5 ,-2.2) {\scriptsize Note a presença do \emph{a}};
}
\schemestop

\end{column}
\end{columns}

\vspace{2cm}

\schemestart 
\chemname{
\chemfig{H_2C=CH-CH=CH-CH=CH_2}}{\small posição das duplas}
\chemmove{
\node[text width=3cm,blue] at (2.0 ,0) (A) {hexa-1,3,5-trieno};
\draw[|->] (2.4,-.1)--(2.4,-0.8); % Line 1
\node[text width=2.0cm,blue] at (2.8 ,-1.1) (A) {\scriptsize \emph{tri} três e \emph{en} (tripla ligação)};
\draw[|->] (1.7,-.1)--(1.7,-0.5)--(1.0,-.5); % seta dupla
\draw[|->] (1.2, -0.1)--(1.2,-1.9); % seta do A
\node[text width=3cm,blue] at (1.5 ,-2.2) {\scriptsize Note a presença do \emph{a}};
}
\schemestop
\vspace{3cm}
\end{bclogo}
\end{frame}


\section{Ciclanos}
\label{sec:org991a330}

\begin{frame}[label={sec:orgacf9d08}]{Ciclanos}
\begin{myex}{Exemplos}
\begin{center}
\begin{tabular}{llll}
 \chemfig{H_2C-[,,2,1]CH_2-[:120,,1]\mcfabove{C}{\mcfright{H}{_2}}(-[:240]\phantom{C})}  & ou & \chemfig{--[:120](-[:240])} & \alert{ciclopropano}\\[0pt]
 &  & \\[0pt]
 \chemfig{H_2C-[,,2,1]CH_2-[:90,,1,1]CH_2-[:180,,1,2]H_2C(-[:270,,2]\phantom{C})}  & ou & \chemfig{--[:90]-[:180](-[:270])} & \alert{ciclobutano}\\[0pt]
 &  & \\[0pt]
 \chemfig{\mcfbelow{C}{\mcfright{H}{_2}}-[:36,,,1]CH_2-[:108,,1]\mcfabove{C}{\mcfright{H}{_2}}-[:180]\mcfabove{C}{\mcfright{H}{_2}}-[:252,,,2]H_2C(-[:324,,2]\phantom{C})} & ou & \chemfig{-[:36]-[:108]-[:180]-[:252](-[:324])} & \alert{ciclopentano}\\[0pt]
\end{tabular}
\end{center}

\end{myex}
\end{frame}


\section{Ciclenos}
\label{sec:org2b3e4e9}

\begin{frame}[label={sec:org7a11eca}]{Ciclenos}
\begin{myex}{Exemplos}
\begin{center}
\begin{tabular}{llll}
 \chemfig{H_2C-[,,2,1]CH=^[:120,,1]\mcfabove{C}{H}(-[:240]\phantom{C})}  & ou & \chemfig{-=^[:120](-[:240])} & \alert{ciclopropeno}\\[0pt]
 &  & \\[0pt]
 \chemfig{H_2C-[,,2,1]CH=^[:90,,1,1]CH-[:180,,1,2]H_2C(-[:270,,2]\phantom{C})}  & ou & \chemfig{*4(-=--)} & \alert{ciclobuteno}\\[0pt]
 &  & \\[0pt]
 \chemfig{\mcfbelow{C}{H}=^[:36,,,1]CH-[:108,,1]\mcfabove{C}{\mcfright{H}{_2}}-[:180]\mcfabove{C}{\mcfright{H}{_2}}-[:252,,,2]H_2C(-[:324,,2]\phantom{C})}   & ou &  \chemfig{=^[:36]-[:108]-[:180]-[:252](-[:324])}  & \alert{ciclopenteno}\\[0pt]
\end{tabular}
\end{center}

\end{myex}
\end{frame}


\begin{frame}[label={sec:orgebd87c3}]{Outros casos}
Em casos como os seguintes, é necessário localizar as duplas ligações.
\alert{A numeração deve ser feita de modo que as insatuarações sejam representadas com os menores números possíveis}.



\begin{center}
	\chemfig{*6(-=-=--)} \af \af
	\chemfig{*8(--=--=---)}
	\vspace{1cm}

	\chemfig{*6(-=-=-=)}
	\chemmove{
		% cycle hexan
		\node[text width=1cm,blue] at (-2.8 ,3.18) {\scriptsize 1}; % C1
		\node[text width=1cm,blue] at (-2.2 ,2.7) {\scriptsize 2}; % C1
		\node[text width=1cm,blue] at (-2.2 ,2.2) {\scriptsize 3}; % C1
		\node[text width=1cm,blue] at (-2.8 ,1.8) {\scriptsize 4}; % C1
		\node[text width=1cm,blue] at (-3.4 ,2.2) {\scriptsize 5}; % C1
		\node[text width=1cm,blue] at (-3.4 ,2.7) {\scriptsize 6}; % C1
 	 	\node[text width=5cm,col8] at (-1.3 ,5,8) {ciclo-hexa-1,3-dieno}; % C1    		
		% Octan
		\node[text width=1cm,blue] at (2.1 ,3.25) {\scriptsize 1}; % C1
    	\node[text width=1cm,blue] at (2.65 ,3.25) {\scriptsize 2}; % C1
    	\node[text width=1cm,blue] at (3.15 ,2.8) {\scriptsize 3}; % C1
    	\node[text width=1cm,blue] at (3.15 ,2.2) {\scriptsize 4}; % C1
    	\node[text width=1cm,blue] at (2.65 ,1.65) {\scriptsize 5}; % C1
		\node[text width=1cm,blue] at (2.15 ,1.65) {\scriptsize 6}; % C1
    	\node[text width=1cm,blue] at (1.65 ,2.2) {\scriptsize 7}; % C1
    	\node[text width=1cm,blue] at (1.65 ,2.8) {\scriptsize 8}; % C1
 	 	\node[text width=5cm,col8] at (8.8 ,5,8) {ciclo-octa-1,4-dieno}; % C1    		
    	% Benzen
		\node[text width=1cm,blue] at (-.1 ,0.9) {\scriptsize 1}; % C1    		
		\node[text width=1cm,blue] at (.45 ,0.55) {\scriptsize 2}; % C1    		
		\node[text width=1cm,blue] at (.45 ,0.02) {\scriptsize 3}; % C1    	
   		\node[text width=1cm,blue] at (-.1 ,-0.4) {\scriptsize 4}; % C1    	
   		\node[text width=1cm,blue] at (-.67 ,0.02) {\scriptsize 5}; % C1    		
   		\node[text width=1cm,blue] at (-.67 ,0.55) {\scriptsize 6}; % C1 
   	 	\node[text width=5cm,col8] at (0.5 ,-.8) {ciclo-hexa-1,3,5-trieno}; % C1 
   	 	 \node[text width=7.2 cm,black] at (0.2 ,-1.7) {\small (também denominado {\bfseries benzeno}, nome aceito pela IUPAC e muito mais utilizado que o apresentado aqui)}; % C1    		   					       		   					    
	}
\end{center}
\end{frame}



\begin{frame}[label={sec:org7de8d6f}]{Fim da Aula}
\begin{tikzpicture}
\node[graduate,sword, devil, minimum size=1cm]{ \bfseries Bons Estudos !!!!};
\end{tikzpicture}
\begin{center}
\begin{tabular}{ccc}
Download Aula & & Lista de Exercícios \\
 \qrcode[height=2in]{https://mark.nl.tab.digital/s/yWAtd5C8mjKjdQa} & & \qrcode[height=2in]{https://mark.nl.tab.digital/s/6kSsDYwW4icCK9X}\\
 \end{tabular}
 \end{center}
\end{frame}
\end{document}
