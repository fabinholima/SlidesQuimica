% Created 2024-07-21 Sun 18:58
% Intended LaTeX compiler: lualatex
\documentclass[presentation,professionalfonts,smaller,aspectratio=169]{beamer}
                


\makeatletter
 \@ifclassloaded{beamer}{%
  %%% save beamer's `solution' environment as `beamersolution':
  \let\beamersolution\solution
  \let\endbeamersolution\endsolution
  %%% "delete" the `solution' environment:
  \let\solution\relax
  \let\endsolution\relax
}{%
}%
\makeatother
\usepackage[utf8]{inputenc}
\usepackage[T1]{fontenc}
%\usepackage[french]{babel}
\usepackage[portuguese]{babel}

%%%% FONTS




\usepackage{xsim}
\usepackage[most]{tcolorbox}
\usepackage{amssymb}
\usepackage{fontawesome}
\newcounter{paragraph}



\DeclareExerciseEnvironmentTemplate{custom}{%
  \begin{tcolorbox}[boxrule = 0pt]
  \tcbox[on line,colback=teal,colframe=teal,coltext=white,size=small]{%
    \faBook\sffamily\bfseries\
    \XSIMmixedcase{\GetExerciseName}
    \GetExerciseProperty{counter}%
  }\quad
}{\end{tcolorbox}}


\DeclareExerciseEnvironmentTemplate{custom2}{%
  \begin{tcolorbox}[boxrule = 0pt]
  \tcbox[on line,colback=violet,colframe=violet,coltext=white,size=small]{%
    \faToggleOn\sffamily\bfseries\
    \XSIMmixedcase{\GetExerciseName}
    \GetExerciseProperty{counter}%
  }\quad
}{\end{tcolorbox}}




\DeclareExerciseType{test}{
	exercise-env = question ,
	solution-env = answer ,
	exercise-template = custom ,
	solution-template = custom2 ,
	exercise-name	= Exemplo. ,
	exercises-name = Exemplo ,
	solution-name = Solução ,
	solutions-name = Sol. ,
	exercise-heading = \textbf ,
	solution-heading = \textbf
}


\xsimsetup{
  exercise/within = section,
  exercise/the-counter =  \arabic{exercise}, 
%%solution-name = solution,  % used with headings=true
solution/print=false,
%print-collection/print=both,
}





\usepackage{colortbl}
\usepackage[tikz]{bclogo}
\usetikzlibrary{fit,patterns,shadows.blur,shapes,mindmap}
\usetikzlibrary{arrows,calc,arrows.meta,decorations.markings,shapes.symbols}
\usetikzlibrary{decorations.pathreplacing, decorations.pathmorphing,calc,arrows,positioning}
\usepackage{tikzpeople}
\usepackage{qrcode,hyperref}
\usepackage{upgreek}
%\usepackage[version=4]{mhchem}
\usepackage{tabularray}


\NewTblrTheme{fancy}{
\SetTblrStyle{caption-tag}{font=\bfseries}
\SetTblrInner[tblr,longtblr]{rowsep=2.5pt}
\DefTblrTemplate{firsthead, middlehead,lasthead}{default}{} % <---
\DefTblrTemplate{contfoot-text}{normal}{\scriptsize\textit{Continued on the next page}}
\SetTblrTemplate{contfoot-text}{normal}
}






\usepackage{chemfig,chemmacros,elements,chemformula}
\chemsetup{modules={all}}
\chemsetup[redox]{pos=top,roman=false}
\chemsetup[redox]{pos=top}
\chemsetup{redox/sep=.5em}
\chemsetup[redox]{explicit-sign=true}
\NewChemPhase\lqdd{\(\ell\)}
\NewChemPhase\gr{grafite}
\NewChemPhase\reac{reação}
\NewChemState\Enthalpy{symbol=H,superscript=,unit=\kilo\joule}%
\usepackage{siunitx}
\setchemfig{fixed length=false, atom sep=2.5em, arrow offset=6pt, scheme debug=false}%,angle increment=30}
\renewcommand*\printatom[1]{\ensuremath{\mathsf{#1}}} % This line changes the font of the atoms to sans serif
%%%% QRCODE
\usepackage{pdfpages}
\usepackage{mol2chemfig}
\usepackage{subfig,caption}
\usepackage{wrapfig}
\usepackage{enumitem}
\setitemize{label=\usebeamerfont*{itemize item}%
\usebeamercolor[fg]{itemize item}
\usebeamertemplate {itemize item}}
\usepackage{array} % ajust colunm table
\usepackage{cancel}
\usepackage[controls]{animate}
\renewcommand{\CancelColor}{\color{red}}

%%%%%%%%%%%%%%%%%%% CONFIG TCOLORBOX 

\newtcolorbox{mybox}[2][]{boxsep=0.5em,left=0.5em,
colback=blue!5!white, colframe=blue!75!black,
fonttitle=\bfseries\sffamily,
colbacktitle=blue!85!red!60,enhanced,
attach boxed title to top left={yshift=-3mm,xshift=5mm},
title=#2,#1}

\newtcolorbox{myrule}[2][]{boxsep=0.5em,left=0.5em,
colback=green!5!white, colframe=blue!75!black,
fonttitle=\bfseries\sffamily,
colbacktitle=blue!85!red!60,enhanced,
attach boxed title to top left={yshift=-3mm,xshift=5mm},
title=#2,#1}


\newtcolorbox{myex}[2][]{boxsep=0.5em,left=0.5em,
  colback=yellow!5!white, colframe=blue!75!black, 
  fonttitle=\bfseries\sffamily,
  colbacktitle=blue!85!red!60,enhanced,
  attach boxed title to top left={yshift=-3mm,xshift=5mm},
  title=#2,#1}


 \definecolor{col1}{HTML}{FF7878}
 \definecolor{col2}{HTML}{51B5F8}
 \definecolor{col3}{HTML}{68E1AA}
 \definecolor{col4}{HTML}{B869EA}
 \definecolor{col5}{HTML}{FF5500}
 \definecolor{col6}{HTML}{FFF8E7}
 \definecolor{col7}{HTML}{FF9966}
 \definecolor{col8}{HTML}{9400D3}



\definesubmol\nobond{-[,0.2,,,draw=none]\scriptstyle\color{blue}}
\newcommand{\re}{\hspace{-1cm}}
\newcommand{\af}{\hspace{2cm}}

\date{}
% \usetheme{minflat}
\usetheme{minflat}
\author{Fábio Lima}
\date{}
\title{Hidrocarbonetos Ramificados}
\hypersetup{
 pdfauthor={Fábio Lima},
 pdftitle={Hidrocarbonetos Ramificados},
 pdfkeywords={},
 pdfsubject={},
 pdfcreator={Emacs 29.4 (Org mode 9.6.15)}, 
 pdflang={En Portuguese}}
\begin{document}

\begingroup
  \setbeamertemplate{headline}{}
  \maketitle
  \endgroup
\begin{frame}{Sumário}
\tableofcontents
\end{frame}




\section{Hidrocarbonetos Ramificados}
\label{sec:org20f64be}

\begin{frame}[label={sec:org0299100}]{Hidrocarbonetos Ramificados}
\begin{myex}{Ramificação}
\begin{itemize}
\item A expressão \alert{grupos substituintes orgânicos} ou, simplesmente \alert{grupos orgânicos} é usada para designar qualquer grupo de átomos que apareça com freqüência nas moléculas orgânicas.
\end{itemize}

\begin{center}
\begin{tabular}{lll}
\chemfig{-CH_3} &  & \chemfig{-CH_2-CH_3}\\[0pt]
 &  & \\[0pt]
\quad metil &  & \qquad    etil\\[0pt]
\end{tabular}
\end{center}


\end{myex}
\end{frame}


\begin{frame}[allowframebreaks]{Grupos substituintes}
\begin{longtblr}[theme=fancy,
    caption = {Grupos substituintes orgânicos formados por carbono e hidrogênio},
    %note{a} = {It is the first footnote.},
    ]{
        colspec = {c c c c }, colsep = 2mm, hlines = {2pt, white},
        %row{odd} = {azure8}, row{even} = {gray8},
        row{1,3,5,7,10,13,16} = {1.5em,azure3,fg=white,font=\bfseries\sffamily},
        %row{12} = {bg=gray8, font=\bfseries},
        %rowsep=.2cm,        
    }
 \hline
     % Radical & Estrutura & Radical & Estrutura   \\
    \SetCell[c=4]{m,8cm} 1 carbono && \\ 
    & metil &  & \chemfig{-CH_3}  \\ \hline
    \SetCell[c=4]{m,8cm} 2 carbonos && \\   
    & etil & \chemfig{-CH_2-CH_3} \\  \hline 
    \SetCell[c=4]{m,8cm} 3 carbonos &&  \\ 
    propril &  \chemfig{-CH_2-CH_2-CH_3}  & isopropil &  \chemfig{-CH([:-90]-CH_3)-CH_3}  \\ \hline 
    \pagebreak
     \SetCell[c=4]{m,8cm} 4 carbonos && \\ %\hline
     %\pagebreak 
    butil &  \chemfig{-CH_2|{(CH_2)_2}CH_3}  & isobutil &  \chemfig{-CH_2-CH([:-90]-CH_3)-CH_3} \\ \hline 
    \emph{s}-butil (\emph{sec}-butil) &  \chemfig{-CH([:-90]-CH_3)-CH_2CH_3} & \emph{t}-butil (\emph{terc}-butil) & \chemfig{-C([:90]-CH_3)([:-90]-CH_3)-CH_3}
      \\ \hline 
     \pagebreak
    \SetCell[c=4]{m,8cm} 5 carbonos &&& \\
     pentil &  \chemfig{-CH_2|{(CH_2)_3}CH_3} & isopentil &  \chemfig{-CH_2-CH_2-CH([:-90]-CH_3)-CH_3} \\  \hline
     neopentil & \chemfig{-CH_2-C([:-90]-CH_3)([:90]-CH_3)-CH_3}  & \emph{t}-pentil (\emph{terc}-pentil) & \chemfig{-C([:90]-CH_3)([:-90]-CH_3)-CH_2-CH_3}\\  \hline 
      \pagebreak
     \SetCell[c=4,]{m,8cm}Outros grupos & && \\
      vinil ou etenil & \chemfig{-CH=CH_2}   & isopropenil & \chemfig{-C([:-90]-CH_3)=CH_2}\\ \hline
      propenil & \chemfig{-CH=CH-CH_3}  & ali ou propen-2-il & \chemfig{-CH_2-CH=CH_2}\\ \hline 
      \pagebreak
   \SetCell[c=4,]{m,10cm} Aromáticos &&& \\
    fenil & \chemfig{-(*6(-=-=-=))}  & naft-1-il & \chemfig{*6(-=(*6(-=-=(-)--))-=-=)}\\ \hline
    benzil & \chemfig{-CH_2-(*6(-=-=-=))} & naft-2-il & \chemfig{*6(-=(*6(-=-(-)=--))-=-=)}\\
      \hline 
\end{longtblr}
\end{frame}

\begin{frame}[allowframebreaks]{Hidrocarbonetos Ramificados - Cadeia Principal}
\begin{mybox}{Definição}
\begin{itemize}
\item Cadeia principal é a maior seqüência de carbonosque contenha as ligações duplas
e triplas (se houver). Em caso de duas sequencias igualmente longas, é a mais
ramificada. Os carbonos que não fazem parte da cadeia principal pertencem às ramificações.
\end{itemize}

\end{mybox}



%%% 1 exemplo

\begin{bclogo}[logo=\bcinfo, noborder=true, barre=none]{1\textsuperscript{0} Exemplo}
 A cadeia principal é a maior sequencia de carbonos

\vspace{.5cm}
\schemestart
\chemfig{H_3C-CH_2-CH([:-90]-CH_3)-CH_2-CH_2-CH_3}
\arrow(.mid east--.mid west){->}[,1,thick]
\chemfig{@{A}C-C-C([:-90]-@{C}C)-C-C-C{}@{B}}
\chemmove{
  \node[inner sep=2pt,fill=blue,fill opacity=0.2,fit=(A) (B)]{};
  \node[inner sep=2pt,fill=red,fill opacity=0.2,fit=(C)]{}; 
  \node[text width=3cm,blue] at (-1,.5) {Cadeia Principal};
  \node[text width=3cm,red] at (-2,-1.2) {Ramificação};
  }  
\schemestop
\end{bclogo}

%%% 2 exemplo

\begin{bclogo}[logo=\bcinfo, noborder=true, barre=none]{2\textsuperscript{0} Exemplo}
A cadeia principal nem sempre está na horizontal 

\vspace{.5cm}
\schemestart
\chemfig{H_3C-CH([:-90]-CH_2-CH_3)-CH([:-90]-CH_3)-CH([:90]-CH_2-CH_2([:0]-CH_3))-CH_2-CH_3}
\arrow(.mid east--.mid west){->}[,1,thick]
\chemfig{@{J}C-@{A}C([:-90]-C-C@{B})-C([:-90]-@{X}C)-@{Z}C([:90]-C-@{E}C([:0]-C@{H}))-@{Q}C-C@{R}}
\chemmove{
  \node[inner sep=2pt,fill=blue,fill opacity=0.2,fit=(A) (B)]{};
  \node[inner sep=2pt,fill=blue,fill opacity=0.2,fit=(A) (Z)]{};
  \node[inner sep=2pt,fill=blue,fill opacity=0.2,fit=(Z) (E)]{};
  \node[inner sep=2pt,fill=blue,fill opacity=0.2,fit=(E) (H)]{};
  \node[inner sep=2pt,fill=red,fill opacity=0.2,fit=(X)]{}; 
  \node[inner sep=2pt,fill=red,fill opacity=0.2,fit=(J)]{}; 
  \node[inner sep=2pt,fill=red,fill opacity=0.2,fit=(Q) (R)]{}; 
  \node[text width=3cm,blue] at (0.1,1) {Cadeia Principal};
  \node[text width=3cm,red] at (-1,-1.2) {Ramificação};
  }  
\schemestop 


\end{bclogo}

%%% 3 exemplo

 \begin{bclogo}[logo=\bcinfo, noborder=true, barre=none]{3\textsuperscript{0} Exemplo}
  No caso de duas ou mais sequências igualmente longas, a cadeia principal é a mais ramificada

  \vspace{.3cm}
  \schemestart
  \chemfig{H_3C-C([:90]-CH_3)([:-90]-CH_3)-CH([:-90]-CH([:0]-CH_3)-CH_3)-CH_2-CH_3}
  \hspace{2cm}
  \chemfig{@{A}C-C([:90]-@{G}C)([:-90]-@{H}C)-@{B}C@{X}([:-90]-C([:0]-@{F}C)-@{Z}C)-@{Q}C-C@{R}}
  \chemmove{
  %\node[text width=3cm,blue] at (2.0 ,0) (A) {buta-1,2-dieno};
  \node[inner sep=2pt,fill=blue,fill opacity=0.2,fit=(A) (B)]{};
  \node[inner sep=2pt,fill=blue,fill opacity=0.2,fit=(X) (Z)]{};
  \node[inner sep=2pt,fill=red,fill opacity=0.2,fit=(Q) (R)]{};
  \node[inner sep=2pt,fill=red,fill opacity=0.2,fit=(F)]{};
  \node[inner sep=2pt,fill=red,fill opacity=0.2,fit=(G)]{};
  \node[inner sep=2pt,fill=red,fill opacity=0.2,fit=(H)]{};
  \node[text width=3cm,blue] at (-.5,0.5) {\scriptsize cadeia principal};
  \node[text width=3cm,red] at (1.8,-.1) {\scriptsize 4  Ramificações};
  }
  \schemestop
  
  \vspace{.3cm}
  
   \schemestart
   \chemfig{@{A}C-C([:90]-@{F}C)([:-90]-@{G}C)-C([:-90]-@{X}C([:0]-C@{W})-C@{Y})-C-@{B}C}
   \chemmove{
   \node[inner sep=2pt,fill=black,fill opacity=0.2,fit=(A) (B)]{};
   \node[inner sep=2pt,fill=red,fill opacity=0.2,fit=(X) (Y)]{};
   \node[inner sep=2pt,fill=red,fill opacity=0.2,fit=(X) (W)]{}; 
   \node[inner sep=2pt,fill=red,fill opacity=0.2,fit=(F)]{};
   \node[inner sep=2pt,fill=red,fill opacity=0.2,fit=(G)]{};
   \node[text width=3.5cm,black] at (0,.5) {\scriptsize Não é cadeia principal};
   \node[text width=3cm,red] at (-1,-2) {\scriptsize 3 Ramificações};
    }
   \schemestop
   %%%
   \hspace{3.5cm}
   \schemestart
   \chemfig{@{D}C-@{A}C([:90]-@{E}C)([:-90]-C@{B})-@{J}C([:-90]-C([:0]-@{F}C)-C@{M})-C-C@{K}}
   \chemmove{
   \node[inner sep=2pt,fill=black,fill opacity=0.2,fit=(J) (K)]{};
   \node[inner sep=2pt,fill=black,fill opacity=0.2,fit=(J) (M)]{};
   \node[inner sep=2pt,fill=red,fill opacity=0.2,fit=(F)]{};
   \node[inner sep=2pt,fill=red,fill opacity=0.2,fit=(A) (B)]{};
   \node[inner sep=2pt,fill=red,fill opacity=0.2,fit=(A) (D)]{};
   \node[inner sep=2pt,fill=red,fill opacity=0.2,fit=(A) (E)]{};
   \node[text width=4.5cm,black] at (2.5,0) {\scriptsize Não é a cadeia principal};
   \node[text width=3cm,red] at (-3,-0.5) {\scriptsize 2 Ramificações};  
   }
   \schemestop 
 
\end{bclogo}

%%% 4 exemplo
 

\begin{bclogo}[logo=\bcinfo, noborder=true, barre=none]{4\textsuperscript{0} Exemplo}
\begin{itemize}
\item Podem existir duas ou mais cadeias equivalentes, neste caso:
\end{itemize}
\vspace{.3cm}
\schemestart
\chemfig{H_3C-CH([:-90]-CH_3)-CH_2-CH_3}
\arrow(nph.mid east--.south west){->}[45]
\chemfig{@{A}C-C([:-90]-@{W}C)-C-C@{B}}
\chemmove{
\node[inner sep=2pt,fill=blue,fill opacity=0.2,fit=(A) (B)]{};
\node[inner sep=2pt,fill=red,fill opacity=0.2,fit=(W)]{}; 
\node[text width=3cm,blue] at (-.5,.5) {\small cadeia principal};
\node[text width=3cm,red] at (0.3,-.7) {\small ramificação};
}
\arrow(@nph.mid east--.north west){->}[-45]
\chemfig{@{Z}C-@{D}C([:-90]-C@{F})-C-C@{E}}
\chemmove{
\node[inner sep=2pt,fill=blue,fill opacity=0.2,fit=(D) (E)]{};
\node[inner sep=2pt,fill=blue,fill opacity=0.2,fit=(D) (F)]{};
\node[inner sep=2pt,fill=red,fill opacity=0.2,fit=(Z)]{};
\node[text width=3cm,blue] at (-.5,.5) {\small cadeia principal};
\node[text width=3cm,red] at (-2,-.5) {\small ramificação};
%  
}
\schemestop
\end{bclogo}



%%% 5 exemplo

\begin{bclogo}[logo=\bcinfo, noborder=true, barre=none]{5\textsuperscript{0} Exemplo}

\schemestart
\chemfig{H_2C=CH-CH([:-90]-CH_2-CH_2-CH_3)-CH([:-90]-CH_3)-CH_2-CH_3}
\arrow(.mid east--.mid west){->}[,1,thick]
\chemfig{@{A}C=C-C([:-90]-@{G}C-C-C@{H})-C([:-90]-@{F}C)-C-C@{B}}
\chemmove{
\node[inner sep=2pt,fill=blue,fill opacity=0.2,fit=(A) (B)]{};
\node[inner sep=2pt,fill=red,fill opacity=0.2,fit=(F)]{};
\node[inner sep=2pt,fill=red,fill opacity=0.2,fit=(G) (H)]{};
\node[text width=3.0cm,blue] at (-1.0,0.5) {cadeia principal};      
}
\schemestop 

\vspace{1.2cm}

\schemestart
\chemfig{C=C-@{B}C([:-90]-C-C-C@{E})-C([:-90]-C)-C-C@{A}}
\chemmove{
\node[inner sep=2pt,fill=black,fill opacity=0.2,fit=(A) (B)]{};
\node[inner sep=2pt,fill=black,fill opacity=0.2,fit=(B) (E)]{};
\node[text width=5.0cm,black] at (-2,-2.5) {\scriptsize Não é a cadeia principal pois, apesar de ser a mais longa não contém a \emph{dupla}};
}
\schemestop 
\hspace{3.5cm}
\schemestart
\chemfig{@{A}C=C-@{B}C([:-90]-C-C-C@{E})-@{G}C([:-90]-C@{H})-C-C@{J}}
\chemmove{
\node[inner sep=2pt,fill=black,fill opacity=0.2,fit=(A) (B)]{};
\node[inner sep=2pt,fill=black,fill opacity=0.2,fit=(B) (E)]{};
\node[inner sep=2pt,fill=red,fill opacity=0.2,fit=(G) (H)]{};
\node[inner sep=2pt,fill=red,fill opacity=0.2,fit=(G) (J)]{};      
\node[text width=5.0cm,black] at (-5,-.5) {\scriptsize Não é a cadeia principal pois, apesar de incluir a \emph{dupla} e ter o mesmo comprimento da cadeia principal, é menos ramificada};
\node[text width=3.0cm,red] at (0.5,-.5) {1 ramificação};
}
\schemestop 

\end{bclogo}
\end{frame}


\section{Nomeclatura}
\label{sec:org30d9885}

\begin{frame}[label={sec:org8a6affc}]{Nomenclatura}
\begin{myrule}{Regras}

\begin{itemize}
\item Localize a cadeia principal.

\item Numere os carbonos da cadeia principal. Para decidir por qual extremidade deve começar a numeração, baseia-se nos seguintes critérios:

\item Se a cadeia for \alert{insaturada}, comece pela extremidade que apresente \alert{insaturação} mais próxima a ela.

\item Se a cadeia for \alert{saturada}, comece pela extremidade que tenha uma \alert{ramificação} mais próxima a ela.

\item Escreva o número de localização da ramificação e, a seguir, separando com um hífen, o nome do grupo orgânico que corresponde à ramificação.

\item Finalmente, escreva o nome do hidrocarboneto correspondente à cadeia principal, separando-o do nome da ramificação por um hífen
\end{itemize}

\end{myrule}
\end{frame}


\section{Nomenclatura Exemplos}
\label{sec:org9fcd756}

\begin{frame}[allowframebreaks]{Nomenclatura Exemplos}
%% 1
\begin{myrule}{Exemplo 1}
\vspace{1cm}
\schemestart
\chemname{\chemfig{H_3\mcfabove{C}{\mcfatomno{5}}-\mcfabove{C}{\mcfatomno{4}}H_2-\mcfabove{C}{\mcfatomno{3}}H_2-\mcfabove{C}{\mcfatomno{2}}H([:-90]-CH_3)-\mcfabove{C}{\mcfatomno{1}}H_3}}{\alert{ 2-metil-pentano}}
\chemmove{
\node[text width=6cm,black] at (4,-1) {4-metil-pentano está incorreto};  
\node[ellipse callout,rounded corners,fill=col7,callout absolute pointer={(.5,0)}, callout pointer width=1cm] at ([shift={(.5cm,1cm)}]3,0) {Extremidade mais próxima da ramificação};
    }
\schemestop
\end{myrule}

%%% 2
\begin{myrule}{Exemplo 2}
\vspace{1cm}
\schemestart
\chemname{
\chemfig{H_3\mcfabove{C}{\mcfatomno{1}}-\mcfabove{C}{\mcfatomno{2}}H_2-\mcfabove{C}{\mcfatomno{3}}H([:-90]-CH_3)-\mcfabove{C}{\mcfatomno{4}}H_2-\mcfabove{C}{\mcfatomno{5}}H_2-\mcfabove{C}{\mcfatomno{6}}H_3}}{\alert{3-metil-hexano}}
\chemmove{
\node[text width=6cm,black] at (4,-1) {4-metil-hexano está incorreto};  
\node[ellipse callout,rounded corners,fill=col7,callout absolute pointer={(.5,0)}, callout pointer width=1cm] at ([shift={(.5cm,1cm)}]3,0) {Extremidade mais próxima da ramificação};
    }
\schemestop
\end{myrule}

%% 3
\begin{myrule}{Exemplo 3}
\vspace{1cm}
\schemestart
\chemname{
\chemfig{H_3\mcfabove{C}{\mcfatomno{1}}-\mcfabove{C}{\mcfatomno{2}}H_2-\mcfabove{C}{\mcfatomno{3}}H([:-90]-CH_3)-\mcfabove{C}{\mcfatomno{4}}H_2-\mcfabove{C}{\mcfatomno{5}}H_3}}{\alert{3-metil-pentano}}
\chemmove{
%\node[text width=6cm,black] at (4,-1) {4-metil-hexano está incorreto};  
\node[ellipse callout,rounded corners,fill=col7,callout absolute pointer={(.5,0)}, callout pointer width=1cm] at ([shift={(.5cm,1cm)}]3,0) {A numeração pode ser em qualquer sentido};
    }
\schemestop
\end{myrule}
%% 4
\begin{myrule}{Exemplo 4}
Se houver mais de um substituinte, deve-se numerar a cadeia principal começando pela extremidade da qual haja uma ramificação mais próxima.

\vspace{1.5cm}
\schemestart
\chemname{
\chemfig{H_3\mcfabove{C}{\mcfatomno{1}}-\mcfabove{C}{\mcfatomno{2}}H([:-90]-CH_3)-\mcfabove{C}{\mcfatomno{3}}H([:-90]-CH_3)-\mcfabove{C}{\mcfatomno{4}}H_2-\mcfabove{C}{\mcfatomno{5}}H_3}}{\alert{2,3-dimetil-pentano}}
\chemmove{
\node[text width=5cm,ellipse callout,rounded corners,fill=col7,callout absolute pointer={(.5,0)}, callout pointer width=1cm] at ([shift={(.5cm,1cm)}]3,0) {Segue e menor numeração para o radicais usar vírgula para ponto e hífen para os nomes};
    }
\schemestop
\end{myrule}

%%% 5
\begin{myrule}{Exemplo 5}
\vspace{1cm}
\schemestart
\chemfig{H_3\mcfabove{C}{\mcfatomno{1}}-\mcfabove{C}{\mcfatomno{\hspace{.2cm}2}}([:-90]-CH_3)([:90]-CH_3)-\mcfabove{C}{\mcfatomno{3}}H_2-\mcfabove{C}{\mcfatomno{4}}H_2-\mcfabove{C}{\mcfatomno{5}}H_3}
\chemmove{
 \node[text width=3cm,black] at (3.4 ,0) (A) {\alert{2,2-dimetil-pentano}};
 \draw[|->] (2.1,-.1)--(2.1,-0.8); % Line 1
\node[text width=6cm,black] at (4,-1.3) {Note a repetição da númeração \\  use \emph{di} para indicar dois radicais idênticos}; 
    }
\schemestop
\vspace{1cm}
\end{myrule}

%%% 6
\begin{myrule}{Exemplo 6}
\vspace{1cm}
\schemestart
\chemfig{H_3\mcfabove{C}{\mcfatomno{1}}-\mcfabove{C}{\mcfatomno{\hspace{.2cm}2}}([:-90]-CH_3)([:90]-CH_3)-\mcfabove{C}{\mcfatomno{3}}H([:-90]-CH_3)-\mcfabove{C}{\mcfatomno{4}}H_2-\mcfabove{C}{\mcfatomno{5}}H_3}
\chemmove{
 \node[text width=7cm,black] at (4.4 ,0) (A) {\alert{2,2,3-dimetil-pentano}};
 \node[text width=7cm,black] at (4.4 ,-1) (A) {(3,3,4-trimetil-pentano está incorreto)};
}
\schemestop
\end{myrule}
%%%%% 7 
\begin{myrule}{Exemplo 7}
\vspace{1cm}
\schemestart
\chemfig{(!\nobond\chemabove[1.2ex]{}{1}{})(!\nobond\chemabove[3.9ex]{}{4}{})CH_3-(!\nobond\chemabove[1.2ex]{}{2}{})(!\nobond\chemabove[3.9ex]{}{3}{})CH([:-90]-CH_3)-(!\nobond\chemabove[1.2ex]{}{3}{})(!\nobond\chemabove[3.9ex]{}{2}{})CH([:-90]-CH_3)-(!\nobond\chemabove[1.2ex]{}{4}{})(!\nobond\chemabove[3.9ex]{}{1}{})CH_3}
\chemmove{
\node[text width=3cm,black] at (-1 ,-1.5) (A) {\alert{2,3-dimetil-butano}};
\node[text width=5cm, ellipse callout,rounded corners,fill=col7,callout absolute pointer={(.5,0)}, callout pointer width=1cm] at ([shift={(.5cm,1cm)}]3,0) {Ambas as numerações são equivalentes};
    }
\schemestop
\vspace{1.4cm}
\end{myrule}

%%%% 8 

\begin{myrule}{Exemplo 8}

\vspace{1cm}
\schemestart
\chemfig{H_3C-CH([:-90]-CH_2-CH_3)-CH_2-CH([:90]-CH_2-CH_2-CH_3)-CH_2-CH_3}
\chemmove{
%% Enumerate cadeia
\node[text width=1cm,blue] at (-4.7 ,-1.4) (A) {\scriptsize 1}; % C1
\node[text width=1cm,blue] at (-4.7 ,-.6) (A) {\scriptsize 2}; % C2
\node[text width=1cm,blue] at (-4.3 ,0.4) (A) {\scriptsize 3}; % C3
\node[text width=1cm,blue] at (-3.3 ,0.4) (A) {\scriptsize 4}; % C4
\node[text width=1cm,blue] at (-2.5 ,0.3) (A) {\scriptsize 5}; % C5
\node[text width=1cm,blue] at (-2.5 ,0.9) (A) {\scriptsize 6}; % C6
\node[text width=1cm,blue] at (-2.5 ,1.5) (A) {\scriptsize 7}; % C7
\node[text width=1cm,blue] at (-2.5 ,2.3) (A) {\scriptsize 8}; % C8
%%% Fim numera cadeia
\node[text width=9cm,black] at (5.6 ,1) (A) {\alert{5-{\bfseries{\color{black}{e}}}til-3-{\bfseries{\color{black}{m}}}etil-octano}};
%% Seta Nome
\draw[<-] (1.5, 0.8)--(1.5,0)--(4,0); % seta do e
\draw[<-] (2.4, 0.8)--(2.4,0); % seta do m
\node[text width=3cm,black] at (5.8 ,-0) {ordem alfabética: ``e'' vem antes de ``m''};
\node[text width=3cm, ellipse callout,rounded corners,fill=col7,callout absolute pointer={(-4.5,-1)}, callout pointer width=1cm] at ([shift={(.5cm,1cm)}]0,-3) {\small Extremidade que tem a ramificação mais próxima};
    }
\schemestop
\end{myrule}

%%%% 9
\begin{myrule}{Exemplo 9}
\vspace{1cm}
\schemestart
\chemfig{H_3C-CH([:-90]-CH_2-CH_3)-CH([:90]-CH_3)-C([:-90]-CH([:0]-CH_3)-CH_3)([:90]-CH_2-CH_2-CH_3)-CH_2-CH_3}
\chemmove{
%% Enumerate cadeia
\node[text width=1cm,blue] at (-4.5 ,-1.4) (A) {\scriptsize 1}; % C1
\node[text width=1cm,blue] at (-4.5 ,-.6) (A) {\scriptsize 2}; % C2
\node[text width=1cm,blue] at (-4.3 ,0.4) (A) {\scriptsize 3}; % C3
\node[text width=1cm,blue] at (-3.3 ,0.4) (A) {\scriptsize 4}; % C4
\node[text width=1cm,blue] at (-2.5 ,0.3) (A) {\scriptsize 5}; % C5
\node[text width=1cm,blue] at (-2.5 ,0.9) (A) {\scriptsize 6}; % C6
\node[text width=1cm,blue] at (-2.5 ,1.5) (A) {\scriptsize 7}; % C7
\node[text width=1cm,blue] at (-2.5 ,2.3) (A) {\scriptsize 8}; % C8
%%% Fim numera cadeia
\node[text width=9cm,black] at (5.6 ,1) (A) {\alert{5-{\bfseries{\color{black}{e}}}til-5-{\bfseries{\color{black}{i}}}spopropil-3,4-di{\bfseries{\color{black}{m}}}etil-octano}};
%% Seta Nome
\draw[<-] (1.5, 0.8)--(1.5,0)--(6,0)--(6,-0.5); % seta do e
\draw[<-] (2.3, 0.8)--(2.3,0); % seta do i
\draw[<-] (4.8, 0.8)--(4.8,0); % seta do m
\node[text width=4cm,black] at (5.2 ,-.9) {ordem alfabética: ``e'' vem antes de ``i'' que vem antes de ``m''};
    }
\schemestop
\end{myrule}



\begin{myrule}{Exemplo 10}
\vspace{.5cm}
 \schemestart
 \chemfig{@{A}H_3\mcfabove{C}{\mcfatomno{4}}-\mcfabove{C}{\mcfatomno{3}}H([:-90]-CH_3)-\mcfabove{C}{\mcfatomno{2}}H=\mcfabove{C}{\mcfatomno{1}}H_2@{B}{}}
 \chemmove{
 \node[inner sep=2pt,fill=blue,fill opacity=0.2,fit=(A) (B)]{};
 \node[text width=4cm,red] at (7.7 ,0) {3-metil-but-1-eno};
 \draw[<-] (5.8, -.2)--(5.8, -1);
 \node[text width=.5cm,black] at (5.3 ,-1.3) {localiza};
 \node[text width=3cm,black] at (5.8 ,-1.7) {a ramificação};
 \draw[<-] (7.6, -.2)--(7.6, -1);
 \node[text width=.5cm,black] at (7.4 ,-1.3) {localiza};
 \node[text width=3cm,black] at (8.8 ,-1.7) {a insaturação};
 \node[text width=4cm, ellipse callout,rounded corners,fill=col7,callout absolute pointer={(-.8,-.05)}, callout pointer width=1cm] at ([shift={(.5cm,1cm)}]0,-3) {\small Extremidade mais próxima da insaturação};
}
\schemestop
\vspace{2cm}
%
\end{myrule}
%%%% Ex. 11
\begin{myrule}{Exemplo 11}
 \vspace{.5cm}
 \schemestart
 \chemfig{@{A}\mcfabove{C}{\mcfatomno{5}}H_3-\mcfabove{C}{\mcfatomno{4}}H_2-\mcfabove{C}{\mcfatomno{3}}H([:-90]-CH_3)-\mcfabove{C}{\mcfatomno{2}}H=\mcfabove{C}{\mcfatomno{1}}H_2@{B}{}}
 \chemmove{
 \node[inner sep=2pt,fill=blue,fill opacity=0.2,fit=(A) (B)]{};
 \node[text width=4cm,red] at (5.7 ,0) {3-metil-pent-1-eno};
 \draw[<-] (3.8, -.2)--(3.8, -1);
 \node[text width=.5cm,black] at (3.5 ,-1.3) {localiza};
 \node[text width=3cm,black] at (4.1 ,-1.7) {a ramificação};
 \draw[<-] (5.7, -.2)--(5.7, -1);
 \node[text width=.5cm,black] at (5.5 ,-1.3) {localiza};
 \node[text width=3cm,black] at (6.5 ,-1.7) {a insaturação};
 \node[text width=2cm, ellipse callout,rounded corners,fill=col7,callout absolute pointer={(-.8,-.05)}, callout pointer width=1cm] at ([shift={(.5cm,1cm)}]0,-3) {\small Extremidade mais próxima da insaturação};
}
\schemestop
\vspace{3cm}
\end{myrule}


%%% Ex 12

\begin{myrule}{Exemplo 12}
 \vspace{.5cm}
 \schemestart
 \chemfig{@{A}\mcfabove{C}{\mcfatomno{5}}H_3-\mcfabove{C}{\mcfatomno{4}}H([:-90]-CH_3)-\mcfabove{C}{\mcfatomno{3}}H([:-90]-CH_3)-\mcfabove{C}{\mcfatomno{2}}H=\mcfabove{C}{\mcfatomno{1}}H_2@{B}{}}
 \chemmove{
 \node[inner sep=2pt,fill=blue,fill opacity=0.2,fit=(A) (B)]{};
 \node[text width=4cm,red] at (3.7 ,0) {3,4-dimetil-pent-1-eno};
 %\draw[<-] (3.8, -.2)--(3.8, -1);
 %\node[text width=.5cm,black] at (3.5 ,-1.3) {localiza};
 %\node[text width=3cm,black] at (4.1 ,-1.7) {a ramificação};
 %\draw[<-] (5.7, -.2)--(5.7, -1);
 %\node[text width=.5cm,black] at (5.5 ,-1.3) {localiza};
 %\node[text width=3cm,black] at (6.5 ,-1.7) {a insaturação};
 \node[text width=2cm, ellipse callout,rounded corners,fill=col7,callout absolute pointer={(-.8,-.05)}, callout pointer width=1cm] at ([shift={(.5cm,1cm)}]0,-3) {\small Extremidade mais próxima da insaturação};
}
\schemestop
\vspace{2cm}
\end{myrule}


%%% Ex 13

\begin{myrule}{Exemplo 13}
 \vspace{.5cm}
 \schemestart
 \hspace{4cm}\chemfig{@{A}H_3\mcfabove{C}{\mcfatomno{1}}-\mcfabove{C}{\mcfatomno{2}}~\mcfabove{C}{\mcfatomno{3}}-\mcfabove{C}{\mcfatomno{4}}H_2-\mcfabove{C}{\mcfatomno{5}}H([:-90]-CH_3)-\mcfabove{C}{\mcfatomno{6}}H_3@{B}}
 \chemmove{
 \node[inner sep=2pt,fill=blue,fill opacity=0.2,fit=(A) (B)]{};
 \node[text width=8cm,red] at (5,0){5-metil-hex-2-ino};
 \node[text width=2cm, ellipse callout,rounded corners,fill=col7,callout absolute pointer={(-4,-.05)}, callout pointer width=1cm] at ([shift={(.5cm,1cm)}]-8,-1.5) {\small Extremidade mais próxima da insaturação};
}
\schemestop
\vspace{.5cm}
\end{myrule}

\begin{myrule}{Exemplo 14}
 \vspace{.5cm}
 \schemestart
 \hspace{4cm}\chemfig{@{A}H_3\mcfabove{C}{\mcfatomno{1}}-\mcfabove{C}{\mcfatomno{2}}~\mcfabove{C}{\mcfatomno{3}}-\mcfabove{C}{\mcfatomno{4}}H([:-90]-CH_3)-\mcfabove{C}{\mcfatomno{5}}H([:-90]-CH_3)-\mcfabove{C}{\mcfatomno{6}}H_3@{B}}
 \chemmove{
 \node[inner sep=2pt,fill=blue,fill opacity=0.2,fit=(A) (B)]{};
 \node[text width=8cm,red] at (5,0){4,5-dimetil-hex-2-ino};
 \node[text width=2cm, ellipse callout,rounded corners,fill=col7,callout absolute pointer={(-4,-.05)}, callout pointer width=1cm] at ([shift={(.5cm,1cm)}]-8,-1.5) {\small Extremidade mais próxima da insaturação};
}
\schemestop
\vspace{.5cm}
\end{myrule}


 \begin{myrule}{Exemplo 15}
 \vspace{.5cm}
 \schemestart
 \chemfig{H_3C-C([:90]=CH_2)-CH([:-90]-CH_2-CH_3)-CH_3}
 \chemmove{
 \node[text width=1cm,blue] at (-2.1 ,.8) {\scriptsize 1}; % C1
 \node[text width=1cm,blue] at (-2.1 ,.2) {\scriptsize 2}; % C2
 \node[text width=1cm,blue] at (-1.2 ,.3) {\scriptsize 3}; % C3
 \node[text width=1cm,blue] at (-1.4 ,-.6) {\scriptsize 4}; % C4
 \node[text width=1cm,blue] at (-1.4 ,-1.3) {\scriptsize 5}; % C5
 \node[text width=8cm,red] at (5,0){2,3-dimetil-pent-1-eno};
}
 \schemestop
 \end{myrule}
\end{frame}


\section{Hidrocarbonetos cadeia mista}
\label{sec:org710f3bd}

\begin{frame}[label={sec:orgfc7e749}]{Hidrocarbonetos cadeia mista}
\begin{itemize}
\item Quando um hidrocarboneto possui cadeia mista, a nomenclatura é semelhante as cadeias ramificadas abertas. Veja os exemplos.
\end{itemize}

\begin{center}
\chemname{\chemfig{*5(--(-CH_3)---)}}{metil-ciclo-pentano}\af
\chemname{\chemfig{**6(----(-CH_3)--)}}{metil-benzeno} \af 
\chemname{\chemfig{**6(--(!\nobond\chembelow[0.5ex]{}{3}{})(-CH_3)-(!\nobond\chemabove[1.4ex]{2}{})-(!\nobond\chemabove[0.2ex]{}{\quad 1})(-CH_3)--)}}{1-3-dimetil-benzeno}
 \end{center}
\end{frame}

\begin{frame}[label={sec:orgeb8a570}]{Hidrocarbonetos cadeia mista}
\begin{itemize}
\item Quando há dois substituintes diferentes, eles devem ser citados em \alert{ordem alfabética. O número 1 é dado ao subtituinte citado primeiro} de acordo com a ordem alfabética.
\end{itemize}


\schemestart
\chemfig{**6(---(-CH_3)-(-CH_2([:0]-CH_3))--)} \af 
\chemfig{*5(--(-CH_2-CH_3)--(-H_3C)-)} \af
\chemfig{CH_2([:90]-(*6(---(-CH_3)---)))-CH_2-CH_3}
\chemmove{
	\node[text width=1cm,blue] at (-12.8 ,.8) {\scriptsize 1}; % C1	
	\node[text width=1cm,blue] at (-12.3, .38) {\scriptsize 2}; % C1
	\node[text width=5cm,black] at (-11.5 ,-1) {1-\alert{e}til-2-\alert{m}etil-benzeno}; % C1	
	\node[text width=1cm,blue] at (-8.2 ,.27) {\scriptsize 1}; % C1		
	\node[text width=1cm,blue] at (-7.57 ,.72) {\scriptsize 2}; % C1
	\node[text width=1cm,blue] at (-7.25 ,.01) {\scriptsize 3}; % C1
	\node[text width=5cm,black] at (-7.3 ,-1) {1-\alert{e}til-3-\alert{m}etil-ciclo-pentano}; % C1
	\node[text width=1cm,blue] at (-2.35 ,1.8) {\scriptsize 1}; % C1
	\node[text width=1cm,blue] at (-1.9 ,1.56) {\scriptsize 2}; % C1
	\node[text width=1cm,blue] at (-1.9 ,1.0) {\scriptsize 3};
	\node[text width=1cm,blue] at (-2.35 ,0.59) {\scriptsize 4};
	\node[text width=5cm,black] at (-1.7 ,-1) {1-\alert{m}etil-4-\alert{p}ropil-ciclo-hexano}; % C1
	\draw[->] (-13.6,-2.)--(-13.6,-1.1);
	\draw[->] (-12.6,-2.)--(-12.6,-1.1);
  \node[text width=6cm,col8] at (-11.6 ,-2.5) {ordem alfabética: ``e'' antes de ``m''};
  	\draw[->] (-3.8,-2.)--(-3.8,-1.1);
  	\draw[->] (-2.7,-2.)--(-2.7,-1.1);
  \node[text width=6cm,col8] at (-1.7 ,-2.5) {ordem alfabética: ``m'' antes de ``p''};
}
\schemestop
\end{frame}

\begin{frame}[label={sec:orga42fba3}]{}
\begin{itemize}
\item Se houver mais de dois substituintes, eles serão citados em \alert{ordem alfabética. O número 1 deve ser dado ao substituinte que permitir que um segundo substituintes receba o menor número possível}
\end{itemize}

\hspace{.5cm}\chemname{\chemfig{CH_3-[:273,,1]-[:327]-[:255]-[:183]-[:111](-[:39])(-[:223,,,2]H_3C)-[:155,,,2]H_3C}}{1,1,2-trimetil-ciclo-pentano}\af 
\chemname{\chemfig{*6((-CH_2([:180]-H_3C))--(-CH_3)-(-([:0]CH_2-CH_2-CH_3))---)}}{4-etil-2-metil-1-propil-ciclo-hexano}\\
\vspace{.2cm}
\chemname{\chemfig{**6(-(-CH_3)--(-CH_3)-(-CH_3)--)}}{1,2,4-trimetil-benzeno} \af \af 
\chemname{\chemfig{**6(-(-C_2H_5)--(-C_2H_5)-(-CH_3)--)}}{2,4-dietil-1-metil-benzeno}
\chemmove{
	\node[text width=1cm,blue] at (-6.3 ,3.2) {\scriptsize 1}; % C1	
	\node[text width=1cm,blue] at (-5.8 ,3.5) {\scriptsize 2}; % C1	
	%%%
	\node[text width=1cm,blue] at (-0.28 ,4.65) {\scriptsize 1}; % C1
	\node[text width=1cm,blue] at (-0.11 ,3.9) {\scriptsize 2}; % C1
	\node[text width=1cm,blue] at (-0.68 ,3.36) {\scriptsize 3}; % C1
	\node[text width=1cm,blue] at (-1.3 ,3.9) {\scriptsize 4}; % C1
	%%%%
	\node[text width=1cm,blue] at (-7.3 ,0.98) {\scriptsize 1}; % C1
	\node[text width=1cm,blue] at (-6.98 ,0.66) {\scriptsize 2}; % C1
	\node[text width=1cm,blue] at (-6.8 ,0.1) {\scriptsize 3}; % C1
	\node[text width=1cm,blue] at (-7.3 ,-0.4) {\scriptsize 4}; % C1
	%%%
	\node[text width=1cm,blue] at (-1.2 ,0.98) {\scriptsize 1}; % C1
	\node[text width=1cm,blue] at (-0.91 ,0.66) {\scriptsize 2}; % C1
	\node[text width=1cm,blue] at (-0.78 ,0.1) {\scriptsize 3}; % C1
	\node[text width=1cm,blue] at (-1.2 ,-0.4) {\scriptsize 4}; % C1
	\node[text width=4cm,ellipse callout,rounded corners,fill=col7,callout absolute pointer={(-0,0.8)}, callout pointer width=1cm] at ([shift={(.5cm,1cm)}]3,-1) {\small No exemplo ao lado -\ch{C2H5} é uma maneira de representar o grupo etil $\rm -CH_2-CH_3$};
}
\end{frame}

\begin{frame}[label={sec:org8eecaab}]{}
\begin{itemize}
\item Quando uma molécula de benzeno que contém \alert{dois} grupos substituintes ligados ao anel, podemos usar o prefixo \emph{orto}, \emph{meta} e \emph{para}.
\end{itemize}

\begin{bclogo}[logo=\bcattention, noborder=true, barre=none]{Atenção}
Os prefixos \emph{orto}, \emph{meta} e \emph{para} podem ser \alert{utilizados apenas quando um anel benzênico possuir dois grupos ligados a ele ligados:}\\[0pt]

\(\bullet\) \emph{orto} \alert{indica 1, 2}; \quad \(\bullet\) meta \alert{indica 1, 3}; \quad \(\bullet\) \emph{para} \alert{indica 1, 4}
\end{bclogo}

\chemname{\chemfig{**6(---(-CH_3)-(-CH_3)--)}}{\emph{orto}-dimetil-benzeno \\ {\color{col8} \emph{orto:}indica posição 1,2}} \af  
\chemname{\chemfig{**6(--(-CH_3)--(-CH_3)--)}}{\emph{orto}-dimetil-benzeno \\ {\color{col8} \emph{meta:}indica posição 1,3}} \af
\chemname{\chemfig{**6(-(-CH_3)---(-CH_3)--)}}{\emph{orto}-dimetil-benzeno \\ {\color{col8} \emph{para:}indica posição 1,4}} 
\end{frame}


\begin{frame}[label={sec:orge6c4859}]{Cadeia Mista - Nomenclatura trivial}
\begin{itemize}
\item Alguns exemplos de nomes triviais de hidrocarbonetos aromáticos são: \emph{tolueno, orto-xileno, meta-xileno} e \emph{para-xileno}.
\end{itemize}


\chemname{\chemfig{**6(---(-CH_3)---)}}{tolueno} \af  
\chemname{\chemfig{**6(---(-CH_3)-(-CH_3)--)}}{\emph{orto}-xileno} \af  
\chemname{\chemfig{**6(--(-CH_3)--(-CH_3)--)}}{\emph{meta}-xileno} \af
\chemname{\chemfig{**6(-(-CH_3)---(-CH_3)--)}}{\emph{para}-xileno} 
\end{frame}


\begin{frame}[label={sec:orga1a0605}]{Cadeis Mistas - Outros exemplos}
\chemname{\chemfig{**6(---(-CH_3)-(-C_2H_5)--)}}{\emph{orto}-etil-metil-benzeno} \af  
\chemname{\chemfig{**6(--(-CH_3)--(-C_2H_5)--)}}{\emph{meta}-etil-metil-benzeno} \af
\chemname{\chemfig{**6(-(-CH_3)---(-C_2H_5)--)}}{\emph{para}-etil-metil-benzeno} \af
\chemname{\chemfig{**6(-(-C_2H_5)--(-C_2H_5)-(-CH_3)--)}}{2,4-dietil-1-metil-benzeno} 

\begin{itemize}
\item Os prefixos \emph{\alert{orto}}, \emph{\alert{meta}} e \emph{\alert{para}} vêm do grego e podem ser traduzidos, respectivamente, por \emph{''diretamente''}, \emph{``depois de''} e \emph{``mais longe de''}
\end{itemize}
\end{frame}





\begin{frame}[label={sec:orgba0ca18}]{Fim da Aula}
\begin{tikzpicture}
\node[graduate,sword, devil, minimum size=1cm]{ \bfseries Bons Estudos !!!!};
\end{tikzpicture}
\begin{center}
\begin{tabular}{ccc}
Download Aula & & Lista de Exercícios \\
 \qrcode[height=2in]{https://mark.nl.tab.digital/s/yWAtd5C8mjKjdQa} & & \qrcode[height=2in]{https://mark.nl.tab.digital/s/6kSsDYwW4icCK9X}\\
 \end{tabular}
 \end{center}
\end{frame}
\end{document}
